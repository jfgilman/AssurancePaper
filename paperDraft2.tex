\documentclass[12pt]{article}
\usepackage{amsmath, amssymb, fullpage}
\usepackage{array}
\usepackage{graphicx,psfrag,epsf}
\usepackage{enumerate}
\usepackage{caption}
\usepackage{subcaption}


\begin{document}

\title{Bayesian Hierarchical Models for Common Components across Multiple System
Configurations}

\begin{abstract}
    We use a Bayesian hierarchical model to assess the reliability of the Joint
    Light Tactical Vehicle (JLTV), which is a family of vehicles. The proposed
    model effectively combines information across three phases of testing and
    across common vehicle components. The analysis yields estimates of failure
    rates for specific failure modes and vehicles as well as an overall estimate
    of the failure rate for the family of vehicles. We are also able to obtain
    estimates of how well vehicle modifications between test phases improve
    failure rates. In addition to using all data to improve on current
    assessments of reliability and reliability growth, we illustrate how to
    leverage the information learned from the three phases to determine
    appropriate specifications for subsequent testing that will demonstrate if
    the reliability meets a given reliability threshold.
\end{abstract}

\section{Introduction}
Reliability is a high priority in the testing of military systems. A common
interest is {\em reliability growth}, which tracks the change, and ideally
improvement, in the reliability of a system as it moves through testing phases
and undergoes periodic corrective actions. Ideally, plans for future test events
should be based on what is observed in previous testing.

In this paper, we look at combining information across multiple test phases with
the objective of planning a future test. When combining information, there is no
omnibus solution. Rather, models need to be carefully considered and evaluated
to ensure that they accurately reflect the data and the underlying physical
processes. Our analysis is motivated by reliability data from the Joint Light
Tactical Vehicle (JLTV), which is a family of vehicles designed to replace
one-third of the legacy high-mobility multipurpose wheeled vehicle (Humvee)
fleet. There are four unique types of vehicles, and within types there are many
possible variants. The primary mission of the family of vehicles is to provide
ground mobility that is deployable worldwide and capable of operating across the
range of military roles including combat, sustainment, police action,
peacekeeping, and security patrol, in all weather and terrain conditions.

The JLTV has been through a series of test events as it has been developed.
Engineering and Manufacturing Development (EMD) included three phases of
testing, where fixes occurred only during a set Corrective Action Period (CAP)
between each phase. For every vehicle, each failure encountered during testing
was recorded and attributed to a specific failure mode. Failures discovered
during mission execution that result in an abort or termination of a mission in
progress are scored as Operational Mission Failures (OMF), and failures of
mission essential components are scored as Essential Function Failures (EFF).
The EFFs tend to include a large portion of the failure modes that drive
maintenance costs and reduce system availability. While requirements are
typically written in terms of OMFs, because OMFs are by definition EFFs that
occurred at critical times, combining the failures provides a more robust
reliability estimate. Since fixes were delayed to CAPs, analysis generally shows
a distinct jump in the system reliability between test phases.

We propose a model that can effectively combine information across the three
phases of testing and across common vehicle components. We use a Bayesian
hierarchical model to assess the reliability of the family of vehicles. The
model accounts for the commonality across vehicle types and allows for
uncertainty quantification. Inferential objectives from the proposed model
include an estimate of the mean miles between failure (MMBF) for each vehicle,
failure mode, and phase of test, and an assessment of the effectiveness of the
corrections completed in the CAPs. Additionally, because future tests are being
planned, we propose methodology to leverage the information learned from the
three phases to determine the length of testing needed to demonstrate a given
reliability threshold in subsequent testing.

\section{Data}
Eight vehicle prototypes were used for reliability, availability,
and maintainability (RAM) testing. There were four two-seat utility vehicles,
two four-seat general purpose vehicles, a two-seat heavy guns carrier, and a
two-seat close combat weapons carrier. The close combat weapons carrier, one
general purpose, and two utility vehicles were tested in Aberdeen, Maryland,
where it is much colder in the winter months but the terrain is not as harsh,
and the heavy weapons carrier, one general purpose, and two utility vehicles
were tested in Yuma, Arizona, where it is warmer but with more challenging
terrain, to attain testing in different environmental and terrain conditions.
The vehicles were driven roughly 3000 to about 5000 miles in Phase 1, between
4500 and 7500 miles in Phase 2, and between 5000 and 6700 miles in Phase
3.\footnote{The exact mileage has been altered to protect proprietary
information.}

There are many similarities between vehicle types; in fact, the most dissimilar
vehicles have over 80\% common parts. Consequently, some failure modes, such as
brakes and radios, will be common to all eight vehicles. Other failure modes
will be related, but not identical, as with hydraulics, frame, and body.

There are 91 OMFs and 1321 EFFs attributed to 26 failure modes recorded across
all eight vehicles and all three test phases. In any given phase of test, every
vehicle is not guaranteed to have a failure of all 26 failure modes, and each
test phase does not necessarily end with a failure. Therefore, we have 624 right
censored observations, accounting for miles driven without observed failures.

\section{Methodology}

\subsection{Modeling Reliability}
A standard reliability analysis employed by the Department of Defense (DoD) test
community considers each test phase independently and uses the exponential
distribution to model the miles between failure ~\cite{ref1}. The traditional
analysis is overly simplistic, relies on correct modeling assumptions
and ignores valuable information learned about the individual vehicles and their
failure modes.

In this paper we propose an alternative approach using a Bayesian hierarchical
model.  We will begin by introducing a hierarchical structure that models the
relationships in the data across all test phases and incorporate known
similarities between vehicle failure modes.  Next, we will consider different
distributional assumptions and their implications.  We will aslo
discuss model diagnostics to consider when choosing a final model. Then we will
use the JLTV data to illustrate this process and discuss system reliability
results. This will lead us into the next section on using these results to
improve the planning of furture test.

\subsubsection{Model Structure}
In the first phase of testing, we assume the vehicle miles at failure, $y$,
follow a lifetime distribution with an unknown set of parameters that include
a rate parameter $\lambda_{ij}$.
Introducing notation,
\begin{equation}
y_{ijk}\mid\lambda_{ij}\sim f(\lambda_{ij},...), \quad i = 1,2,...,v \quad
j=1,2,...,s \quad k=1,2,...,n_{ij}
\end{equation}
where $y_{ijk}$ are the miles between failure for vehicle $i$ failure mode $j$,
$v$ is the number of vehicles, $s$ is the number of failure modes, and $n_{ij}$
are the number of failures of vehicle $i$ failure mode $j$. The number of
failure modes is assumed fixed and known \textit{a priori}.

The prior distribution on failure rate parameter, $\lambda_{ij}$, depends on
whether failure mode $j$ is considered to be common across vehicles or related
but not identical. For the related failure modes, we place a prior distribution
on the collection of $\lambda_{ij}$; in other words, we assume each vehicle has
a distinct failure rate in failure mode $j$ but they arise from a common
distribution.  If failure mode $j$ is considered common across vehicles, the collection of
failure rates is collapsed to a single parameter, $\lambda_{ij} = \lambda_j$. As
with the related failure modes, a prior distribution is placed on the single
failure rate. The prior distributions are independent across failure mode, and
can potentially have different hyperparameters.

The Phase 1 analysis yields an estimate of the failure rate $\lambda_{ij}$ for
each of the vehicles for failure modes that are related.  We are assuming the
vehicles are conditionally independent, if we also assume the miles at failure
come from an exponential distribution the overall failure rate across all
failure modes can be easily calculated by $\sum_{j}\lambda_{j}$.  The
exponential distribution is commonly used in reliability analysis because of its
convenience, but as we will discuss in later sections it can be unrealistic in
many real-world situations.

\subsubsection{Fix Effectiveness}
After the first CAP, Test Phase 2 begins with the repaired vehicles. To capture
these revisions, the PM2 reliability growth model \cite{EH06} is often used.
This model explicitly captures testing phases, choices about which failure modes
to correct, and the potential of not completely eliminating a failure upon
repair. One of the downsides of PM2 is that many parameters of potential
interest, such as the Fix Effectiveness Factor (FEF), which measures how much
repairs improve failure rates, are typically fixed. A common value for FEF is
0.70. We follow the premise of this type of model, but allow a more flexible and
data-driven result that is less dependent on hard-coded assumptions.

One normal assumption used in reliability growth modeling is nondecreasing
failure rates; that is either the fixes were effective or had no effect, but did
not degrade the family of vehicles.  This should generally be the case, but
because we are dealing with complex systems and testing conditions are rarely
completely consistent, we will sometimes see decreases in failure rates after
adjustments are made.  Therefore for the Phase 2 data, we write the rate
parameters as a function of the rate parameters found in Phase 1. In particular,
we define $\lambda_{ij}^{P2}=(\rho_{j})\lambda_{ij}^{P1}$ where $\rho_{j}$
represents the between phase change in failure mode $j$. Given this definition
of $\lambda_{ij}^{P2}$, we again model the miles to failure for a given vehicle
and failure mode using the same lifetime distribution from Test Phase 1.  We
will choose a prior distribution for the parameter $\rho_{j}$ that has positive
support.  If $\rho_{j}$ is less than one, this represents an improvement in
reliability.  After Phase 2 we can look again at failure rates across failure
modes and vehicles and obtain an overall estimate of the rate for the family of
vehicles.  The analysis of Test Phase 3 follows the same pattern as that shown
in Phase 2. At the end of Phase 3, we can look at failure rates across failure
modes and vehicles and obtain an overall estimate of the rate for the family of
vehicles. Future tests will be planned based on the inferences of Phase 3.

\subsubsection{Distributional Assumption}
The next modeling choice that must be made is slecting an appropriate
probability model This includes a slecting a sampling distribution for the data
and prior distributions for each parameter on which the sampling distribution
depends.  For lifetime data there are a number sampling distribution to choose
from.  Some common choices are exponential, Weibull, lognormal, gamma, inverse
Gaussian, and normal failure time models.  Because all of these distributions
have rate parameter (sometimes called scale = 1/rate) they can be applied to
the proposed modeling structure.  For this paper we will focus on the
exponential and Weibull distributions.

First the exponential model, Test Phase 1,
\begin{equation} f(y_{ijk}|\lambda_{ij})=\lambda_{ij} \exp(-\lambda_{ij}
y_{ijk})
\end{equation}
Test Phase 2,
\begin{equation}
f(y_{ijk}|\lambda_{ij}\rho_{1,j})=\lambda_{ij}\rho_{1,j}
\exp(-\lambda_{ij}\rho_{1,j} y_{ijk})
\end{equation}
Test Phase 3,
\begin{equation}
f(y_{ijk}|\lambda_{ij}\rho1_{j}\rho2_{j})=\lambda_{ij}\rho1_{j}\rho2_{j}
\exp(-\lambda_{ij}\rho1_{j}\rho2_{j} y_{ijk})
\end{equation}

This is by far the most common parametric distribution used in reliability
modeling because of its desirable math mathematical properties and simple
interpretations.  One of these properties is that the failure rate is constant.
For the JLTV setting this would assume that a given component is equally likely
to fail in the first mile of testing as it is in the thousandth.  This may be
reasonable for some reliability settings, but in general we expect vehicle
components to wear down and become more likely to fail the longer they are
driven.

Despite its common uses the assumption of a constant failure rate over time is
rarely justifiable.  It has been well documented the issues that can arise when
this assumption is violated ~\cite{ref2}.  Thus we will also consider the same
hierarchical model structure while using the Weibull distribution.  Now we
assume that the failure times $y_{ijk}$ each follow a $Weibull(\lambda, \gamma)$
distribution with scale parameter $\lambda$ and shape parameter $\gamma$, with
probability density function for Test Phase 1,
\begin{equation}
f(y_{ijk}|\lambda_{ij},\gamma_{i})=\lambda_{ij}\gamma_{i}
y_{ijk}^{\gamma_{i}-1}\exp(-\lambda_{ij} y_{ijk}^{\gamma_{i}}).
\end{equation}
The Weibull distribution is a more flexible model with both a rate parameter
$\lambda_{ij}$ and a shape parameter $\gamma_{j}$.  The exponential is a special
case of the Weibull, when $\gamma = 1$.

It should be noted that we are currently indexing both $\rho$ and $\gamma$, only
by $j$ and not $i$, in other words we are assuming a single shape and between
phase adjustment parameter for each failure mode across vehicles.  All of these
parameters could also be indexed by $i$ and modeled hierarchical or with some
combination of both.  This is something we will omit for this paper but hope to
explore in the future.  One special case we will discuss is when a single
Weibull shape parameter can be assumed across all failure modes.  When this is
justifiable it can greatly simplify test planning calculations, as we will show
in the assurance testing section.  In this paper we will also be assuming on
site repairs return the component to a "good as new" condition and that all
component failures are independent.  That is, when a component such as the
breaks are fixed during a test phase we will treat it as if it is brand  new and
this failure has no effect on the reliability of other components such as the
armor.

\subsubsection{JLTV Reliability}
So far we have discussed a number of different models and the assumptions that
accompany them.  Using McMC we can fit each of these models to the JLTV dataset.
Then we can use the posterior distributions of the systems parameters to
estimate reliability-related quantities of interest for each of the eight
vehicle types.  We are able to look at a number of different statistics related
to reliability at both  the component and system level.  In this study our main
quantity of interest is expected miles to  failure.  In the following figure we
have box plots showing the posterior distributions of  the expected miles to
failure for one of the vehicle types for each of the three phases, first using
the exponential model and second using the more general Weibull model.

\includegraphics[width=6cm, height=5cm]{Boxplots}

The plots show that the posterior distributions from the Weibull model have a
good amount more spread than the exponential.  This is somewhat expected given
the Weibull model is estimating two model parameters to the exponentials one,
but what is concerning is that the center of the distributions is also quite
different between the two models.  All of the Weibull EMTF posterior means
are higher than the exponentials.  This is the first sign that the exponential
model may not be a good fit to the data.

\subsubsection{Model Diagnostics}
As seen in the last section with the JLTV data, the decision of which model to
use can have a drastic impact on reliability assessment and future test
planning.  We will now look at a few  goodness-of-fit methods that can be used
to help in this decision.  The first model selection question we will explore is
the parametric form, exponential versus Weibull.  In statistical modeling we
often face the trade-off between fit and interpretability, and this situation is
no different. Because the Weibull is a more flexible model, it will always, in a
sense, fit the data better, but this comes at a price.  The exponential's
convenient form makes both computation and interpretation straightforward.  When
the Weibull's shape parameter is introduced this advantage is lost.  Thus, when
the overall fit is close to the same between the exponential and the Weibull we
will default to using the exponential.

The first check we used to decide between the exponential and Weibull models is
to fit the Weibull model and look at the posterior distributions of the
shape parameters and determine if one is a reasonable value.  Looking at the
plot below from the JLTV data, we see that for the 26 components it appears
that the value of one falls in the extreme tails of the distributions.  This is
our first clue that the exponential model will not be a good fit to this data.

\includegraphics[width=6cm, height=5cm]{betaPostPlot}

This is not a surprising result, because the exponential model assumes constant
failure rate.  For most vehicle components we would expect the failure rate to
increase as the distance driven grows larger, and this corresponds with a shape
parameter between zero and one.  On the other hand, when the shape parameter is
larger than 1, the failure rate will decrease with time/distance.

Now a more formal diagnostic tool, the Deviance information criterion (DIC).
This is a popular method for comparing the goodness of fit of multiple models.
The DIC method gives a slight penalty for larger numbers of parameters.  The DIC
value is a unitless measure with lower values indicating a better fit to the
data.  In the table below we show the DIC results for the different distribution
and structure combinations we considered with the JLTV dataset.

\begin{tabular}{|l|l|r|}
\multicolumn{3}{c}{\textbf{Goodness of Fit}} \\
\cline{1-3}
Distribution    & Structure & DIC \\
\hline
Exponential   & Single Rate                       & 23258               \\
              & Hierarchical Rate                 & 23022               \\
Weibull       & Single Rate                       & 18750               \\
              & Hierarchical Rate                 & \textbf{18556}      \\
              & Hierarchical Rate (One shape)     & 18677               \\
\hline
\end{tabular}

The single rate structure represents the non-hierarchical model, using one rate
parameter all 8 vehicles for a given failure mode.  The hierarchical rate
structure uses the common gamma distribution for the failure modes that are
common across vehicles.  The final entry in the table is the model that uses one
shape parameter across all 26 failure modes.

While we have shown that the hierarchical Weibull model fits much better than
the exponential, these test still do not tell us that this model is a good fit
for the data.  The last method we will present is called posterior predictive
checking ~\cite{ref3}.  Here we are given important features of the data.
In the JLTV case we were interested in the total failure counts for each phase
and the number of time the miles between failures was less than 140.  We then
used the final model to simulate 5,000 new datasets.  We then plot histograms
for each of the 8 vehicles for all 3 phases.  In Figure 1 are examples of the
histograms produced, with the line showing where the value of the true dataset
fell.  For this method we don’t expect all of the true values to fall in the
center of the distribution.  Values in the tales are to be expected in a random
processes like this one.  We will only be concerned if we find true values
falling in the extreme tails or if we see a common bias to the high or low side
of the distributions.

\begin{figure}
\centering
\begin{subfigure}{.5\textwidth}
  \centering
  \includegraphics[width=.4\linewidth]{Rplot1}
  \caption{Plot 1}
  \label{fig:sub1}
\end{subfigure}%
\begin{subfigure}{.5\textwidth}
  \centering
  \includegraphics[width=.4\linewidth]{Rplot2}
  \caption{Plot 2}
  \label{fig:sub2}
\end{subfigure}
\caption{Posterior Predictive Plots}
\label{fig:test}
\end{figure}

\subsection{Assurance Test Planning}
In this section, we use the resulting reliability model from the previous
section develop a plan for future testing.  Here the objective is to demonstrate
that at a desired level of confidence, the system will meet or exceed a
specified requirement. The methods to be employed will be similar to the
assurance testing as discussed by Hamada et al. ~\cite{ref4}. Bayesian
\emph{reliability assurance tests} are used to ensure that the reliability of
an item meets or exceeds a specified requirement with a desired probability.
Meeker and Escobar ~\cite{ME04} established the difference between this and the
more traditional  \emph{reliability demonstration test}.  The difference being
that the assurance test plan incorporates supplementary data and information,
where the demonstration test only uses test data.

In both test planning methods we are interested in controlling error rates while
minimizing the resources required for testing.  In the DoD acquisition process,
the error rates will be referenced in terms of risk. \emph{Consumer risk} which
considers the event of purchasing a product that does not meet reliability
requirements.  \emph{Producer risk} which considers the event of a product with
acceptable reliability failing a given test and not being considered for
purchase.  Introducing the notation for this section, In the JLTV case study, we
will be determining how many miles on test $T$, are required and how many system
failures to allow $c$, before the product is considered unacceptable.  We will
define $W(t)$ as a random variable that represents the total number of system
failures after $t$ miles.  Suppose that $\pi$ denotes some quantity of interest
that is related to system reliability at a given time. It is common to base both
classical and Bayesian test plans on two specified levels of $\pi$: $\pi_0$, an
\emph{acceptable reliability level} (ARL), and $\pi_1$, a \emph{rejectable
reliability level} (RRL), where $\pi_1 \leq  \pi_0$. Although the precise
definition of ARL and RRL differ between the classical and Bayesian test
criteria, we use them in an equivalent way.

\subsubsection{Classical Approach}
It is quite common to use two criteria in determining classical test plans. The
\emph{producer's risk}, defined as the probability of failing the test when $\pi =
\pi_0$, and the \emph{consumer's risk}, defined as the probability of passing
the test when $\pi = \pi_1$.  Suppose that we specify a maximum value, $\alpha$,
of the producer's risk and a maximum value, $\beta$, of the consumer's risk. For
test planning, these criteria become
$$
\begin{aligned}
	\text{Producer's Risk} &= P \text{(Test Is Failed } \vert \pi_0 \text{)} \\ &=
	P \text{(} W(t) > c \vert \pi_0 \text{)} \leq \alpha
\end{aligned}
$$
and
$$
\begin{aligned}
	\text{Consumer's Risk} &= P \text{(Test Is Passed } \vert \pi_1 \text{)} \\ &=
	P \text{(} W(t) \leq c \vert \pi_1 \text{)} \leq \beta
\end{aligned}
$$
To choose a test plan for specified values of $(\alpha, \pi_0, \beta, \pi_1)$,
we assume a distributional form that defines the relationship between the number
of system failyres $W(t)$ and the reliability $\pi$.  Then we can simply find
the combinations of these probabilities by simultaneously solving these two
equations.  In the case where the component failure times are assumed to be
exponentially distributed with rate parameter $\lambda_i$ the failure counts for
each component come from a homogeneous Poisson process $w_i(t) \sim
\text{Poisson}(\lambda_i t)$, and the system failure counts $W(t) \sim
\text{Poisson}(\lambda_S t)$.  Numerous textbooks provide additional details of
this purely classical approach.

\subsubsection{Bayesian Approach}
Traditional methods for determining the testing procedure only rely on
distributional assumptions and/or asymptotic results. With a Bayesian approach we
incorporate the supplementary data from the previous testing phases with the
hopes of minimizing the resources needed for testing.

We now consider fully Bayesian posterior risks that convey a completely
different outlook from the corresponding classical risks. While the classical
provide assurance that satisfactory devices will pass the test and that
unsatisfactory devices will fail it, posterior risks provide precisely the
assurance that practitioners often desire: if the test is passed, then the
consumer desires a maximum probability $\beta$ that $\pi \leq \pi_1$. On the
other hand, if the test is failed, then the producer desires a maximum
probability $\alpha$ that $\pi \geq \pi_0$. Unlike the average risks, these
posterior risks are fully Bayesian in the sense that they are subjective
probability statements about $\pi$.

For a test that fails, the \emph{posterior producer's risk} is the probability
that $\pi \geq \pi_{0}$, or $P \text{(}\pi \ge \pi_0 \vert \text{Test Is
Failed)}$. Notice that this is simply the posterior probability that $\pi \ge
\pi_0$ given that we have observed more than $c$ failures. In the exponential
case, if we let $\pi$ be the system failure rate $\lambda_S$, then using Bayes'
Theorem, and assuming a maximum allowable posterior producer's risk $\alpha$, an
expression for the posterior producer's risk for the exponential test plan
$(T,c)$ is
$$
\begin{aligned}
    P(\lambda_S \geq \lambda_0 \; \vert \; \text{Test Is Failed)} &= P(\lambda_S
    \geq \lambda_0 \; \vert \; W > c) \\ &= \int_{\lambda_0}^{\infty}
    p(\lambda_S \; \vert W > c) d\lambda_S \\ &= \int_{\lambda_0}^{\infty}
    \frac{f(W > c \vert \lambda_S)}{\int_{0}^{\infty} f(W > c \vert \lambda_S)
    d\lambda_S} d\lambda_S \\ &= \frac{\int_{\lambda_0}^{\infty} [ \sum_{W=0}^c
    \frac{(\lambda_S T)^W exp(-\lambda_S T)}{W!}]p(\lambda_S)d\lambda_S}
    {\int_{0}^{\infty} [ \sum_{W=0}^c \frac{(\lambda_S T)^W exp(-\lambda_S
    T)}{W!}]p(\lambda_S)d\lambda_S} \leq \alpha
\end{aligned}
$$
\\
\\
For simplicity we if fix c to be zero and then perform Monte Carlo integration
using N posterior draws $ \lambda_S^{(j)} $
$$
\begin{aligned}
	 P(\lambda_S \geq \lambda_0 \; \vert \; W = 0) &=
	 \frac{\int_{\lambda_0}^{\infty} exp(-\lambda_S T)p(\lambda_S)d\lambda_S}
	 {\int_{0}^{\infty} exp(-\lambda_S T)p(\lambda_S)d\lambda_S} \\ &\approx
	 \frac{\sum_{j = 1}^{N} exp(-\lambda_S^{(j)} T)I(\lambda_S^{(j)} \geq
	 \lambda_0)} {\sum_{j = 1}^{N} exp(-\lambda_S^{(j)} T)} \end{aligned}
$$

Similarly, given that the test is passed, the \emph{posterior consumer's risk}
is the probability that $\pi \leq \pi_1$, or $P \text{(}\pi \leq \pi_0 \vert
\text{Test Is Passed)}$. Notice that this is simply the posterior probability
that $\pi \leq \pi_1$ given that we have observed no more than $c$ failures. In
the exponential case, if we let $\pi$ be the system failure rate $\lambda_S$,
then using Bayes' Theorem, and assuming a maximum allowable posterior consumer's
risk $\beta$, an expression for the posterior producer's risk for the
exponential test plan $(T,c)$ is

$$
\begin{aligned}
    P(\lambda_S \leq \lambda_1 \; \vert \; \text{Test Is Passed)} &= P(\lambda_S
    \leq \lambda_1 \; \vert \; W \leq c) \\ &= \int_{0}^{\lambda_1} p(\lambda_S \;
    \vert W > c) d\lambda_S \\ &= \int_{0}^{\lambda_1} \frac{f(W > c \vert
    \lambda_S)}{\int_{0}{\infty} f(W > c \vert \lambda_S) d\lambda_S} d\lambda_S \\ &=
    \frac{\int_{0}^{\lambda_1} [ \sum_{W=0}^c \frac{(\lambda_S T)^W
    exp(-\lambda_S T)}{W!}]p(\lambda_S)d\lambda_S} {\int_{0}^{\infty} [
    \sum_{W=0}^c \frac{(\lambda_S T)^W exp(-\lambda_S
    T)}{W!}]p(\lambda_S)d\lambda_S} \leq \beta
\end{aligned}
$$
\\
\\
For simplicity we if fix c to be zero and then perform Monte Carlo integration
using N posterior draws $ \lambda_S^{(j)} $ $$
\begin{aligned}
	 P(\lambda_S \leq \lambda_1 \; \vert \; W = 0) &= \frac{\int_{0}^{\lambda_1}
	 exp(-\lambda_S T)p(\lambda_S)d\lambda_S} {\int_{0}^{\infty} exp(-\lambda_S
	 T)p(\lambda_S)d\lambda_S} \\ &\approx \frac{\sum_{j = 1}^{N}
	 exp(-\lambda_S^{(j)} T)I(\lambda_S^{(j)} \leq \lambda_1)} {\sum_{j = 1}^{N}
	 exp(-\lambda_S^{(j)} T)}
\end{aligned}
$$

\subsubsection{Weibull Case}
In the exponential case, we defined our quantity of interest that is related to
system reliability at a given time $\pi$ to be equal to the system failure rate
$\lambda_S$.  This distributional assumption gave us a convenient distributional
form for our system failure counts with $W(t) \sim \text{Poisson}(\lambda_S t)$
When we assume the miles to failure for each system component follows a Weibull
distribution the system failure counts no longer follow a homogeneous Poisson
process.  Because each component failure rate now varies with time each
component failure count follows a non-homogeneous Poisson process $w_i(t) \sim
\text{Poisson}(\gamma_i \lambda_i t^{\gamma_i})$.  In this case we no longer
have a closed form for the system failure rate $\lambda_S$.  The exception being
if we can assume a single shape parameter $\gamma$ for all components.  In this
case, it can be shown [A1] that the system failure count follows a non-homogeneous
Poisson process $W(t) \sim \text{Poisson}(\gamma \lambda_S t^{\gamma})$.  Now,
we can use the same approach as with the exponential distribution. An expression
for the posterior producer's risk for the exponential test plan $(T,c)$ is
$$
\begin{aligned}
    P(\lambda_S \geq \lambda_0 \; \vert \; \text{Test Is Failed)} &= P(\lambda_S
    \geq \lambda_0 \; \vert \; W > c) \\ &= \int_{\lambda_0}^{\infty}
    p(\lambda_S \; \vert W > c) d\lambda_S \\ &= \int_{0}^{\infty}
    \int_{\lambda_0}^{\infty} \frac{f(W > c \vert \lambda_S)}{\int_{0}^{\infty}
    \int_{0}^{\infty} f(W > c \vert \lambda_S) d\lambda_S d\gamma} d\lambda_S
    d\gamma \\ &= \frac{\int_{0}^{\infty} \int_{\lambda_0}^{\infty}
    [\sum_{W=0}^c \frac{(\gamma \lambda_S T^{\gamma})^W exp(-\gamma \lambda_S
    T^{\gamma})}{W!}]p(\lambda_S)p(\gamma)d\lambda_S d\gamma} {\int_{0}^{\infty}
    \int_{0}^{\infty} [ \sum_{W=0}^c \frac{(\gamma \lambda_S T^{\gamma})^W
    exp(-\gamma \lambda_S T^{\gamma})} {W!}]p(\lambda_S) p(\gamma)
    d\lambda_Sd\gamma} \leq \alpha
\end{aligned}
$$
\\
\\
Once again we can simplify by setting c to be zero and then perform Monte Carlo
integration using N posterior draws for both $\lambda_S^{(j)}$ and
$\gamma^{(j)}$
$$
\begin{aligned}
	 P(\lambda_S \geq \lambda_0 \; \vert \; W = 0) &= \frac{\int_{0}^{\infty}
	 \int_{\lambda_0}^{\infty} exp(- \gamma \lambda_S T^{\gamma})p(\lambda_S)
	 p(\gamma) d\lambda_S d\gamma} {\int_{0}^{\infty} \int_{0}^{\infty} exp(-
	 \gamma \lambda_S T^{\gamma})p(\lambda_S) p(\gamma) d\lambda_S d\gamma} \\
	 &\approx \frac{\sum_{j = 1}^{N} exp(-\gamma^{(j)} \lambda_S^{(j)}
	 T^{\gamma^{(j)}}) I(\lambda_S^{(j)} \geq \lambda_0)} {\sum_{j = 1}^{N}
	 exp(- \gamma^{(j)} \lambda_S^{(j)} T^{\gamma^{(j)}})} \end{aligned}
$$

In the case of multiple shape parameters, these calculations can still be done.
The practitioner must be careful in defining the system reliability quantity of
interest, accounting for each component failure rate.  The computation can
quickly become  unmanagable as the number of components and/or number of allowed
failures becomes large.

It should be noted that in this test planning process you would use the
posterior distribution of $\lambda_S$ from the final developmental testing
phase.  It is also common to incorporate a degradation factor to account for an
expected reduction in reliability from the developmental phase to the
operational phase.  In the Bayesian framework, this entails adding another
multiplicative parameter to the rate parameter with an appropriate prior.   A
10-30 percent reduction from the developmental phase to the operational phase is
common in many DoD applications.

\subsubsection{JLTV Test Plan Results}
The first step in applying this method to our JLTV dataset is defining $\pi_0$,
$\pi_1$, $\alpha$ and $\beta$.  That is the ARL, RRL, acceptable producer risk
and acceptable consumer risk respectively.  With input from subject matter
experts  we will define the producer's target or ARL as the expected number of
failures in the first 140 miles to be less  than 2, and the consumer's minimum
requirements or RRL as the expected number of failures in  the first 80 miles to
greater than 2.  We will use traditional values of 0.1 and 0.05 for $\alpha$ and
$\beta$ respectively, but in practice these values should always be set based on
the true risk tolerances of the consumer and producers.  This gives us the
following probability statements to satisfy when designing the test:
\\
Consumer Risk : $ P(80^\beta (\lambda_S) \geq 2 \; \vert \text{Test Is Passed}) \leq 0.1 $ \\
Producer Risk : $ P(140^\beta (\lambda_S) \leq 2 \; \vert \; \text{Test Is Failed}) \leq 0.05 $

From here we are able to find combinations of failures allowed and test miles
required that satisfy both risk requirements.  The table below show the
resulting combinations for the JLTV data.

\begin{tabular}{|l|l|r|}
\multicolumn{2}{c}{\textbf{JLTV Test Plan}} \\
\cline{1-2}
Failures Allowed    & Test Miles Required \\
\hline
1   & 5,300      \\
2   & 6,500      \\
3   & 7,800      \\
4   & 11,100     \\
5   & 13,100     \\
\hline
\end{tabular}

It should be noted that if $\pi_0$, $\pi_1$, $\alpha$ and $\beta$ are set to
unrealisitc values there can be situations where no combinations of failures
allowed and test miles required will satisfy these probability statements.  In
this paper we have focused on test plans for single units, but this methodology
can be extended to find test plans that allow for a changing number of testing
units.  In this case you would find combinations of failures allowed and test
miles required and number of vehicles tested that satisfy the risk requirements.
More details and examples of this can be found in the assurance testing section
of Hamada et al. ~\cite{ref4}

\subsection{Discussion}

%%%%% ref %%%%%%%%%%%%%%%%%%%%%%%%%%%%%%%%%%%%%%%%%%%%%%
\begin{thebibliography}{99}
\bibitem{ref1} Director, Defense Test and Evaluation. Test and evaluation of
system reliability availability maintainability - A primer; Third Edition
(1982).
\bibitem{ref2} National Research Council. Statistics, Testing, and Defense
Acquisition: Background Papers. \textit{National Academies Press} (1999).
\bibitem{ref3} Gelman, Andrew, Xiao-Li Meng, and Hal Stern. "Posterior
predictive assessment of model fitness via realized discrepancies."
\textit{Statistica sinica} 733-760 (1996).
\bibitem{ref4} M.\ S.\ Hamada, A.\ G.\ Wilson, C.\ S.\ Reese and H.\ F.\ Martz.
Bayesian Reliability. Springer (2008).
\bibitem{ME04} W.\ Q.\ Meeker and L.\ A.\
Escobar. Reliability: the other dimension of quailty. \textit{Quality Technology
and Quantitative Management} 1, 1-25 (2004). \bibitem{HWWGM13} M.\ S.\ Hamada,
A.\ G.\ Wilson, B.\ P.\ Weaver, R.\ W.\ Griffiths and H.\ F.\ Martz. Bayesian
binomial assurance tests for system reliability using component data.
\textit{Journal of Quality Technology} 46, 24-32 (2014). \bibitem{JGHR03} V.\
E.\ Johnson, T.\ L.\ Graves, M.\ S.\ Hamada, and C.\ S.\ Reese. A hierarchical
model for estimating the reliability of complex systems. In Bayesian Statistics,
Vol. 7,\ J.\ M.\ Bernardo, M.\ J.\ Bayerri, J.\ O.\ Berger, A.\ P.\ Dawid, D.\
Heckerman, A.\ F.\ M.\ Smith, and M.\ West eds., Oxford, UK: Oxford University
Press (2003). \bibitem{HMRGJW04} M.\ S.\ Hamada, H.\ F.\ Martz, C.\ S.\ Reese,
T.\ L.\ Graves, V.\ E.\ Johnson, and A.\ G.\ Wilson. A fully Bayesian approach
for combining multilevel failure information in fault tree quantification and
optimal follow-on resource allocation. \textit{Reliability Engineering and
System Safety} 86, 397-405 (2004). \bibitem{EH06} P.\ M.\ Ellner and J.\ B.\
Hall, An approach to reliability growth planning based on failure mode discovery
and correction using AMSAA projection methodology. \textit{IEEE Proceedings of
the Annual Reliability and Maintainability Symposium} (2006).
\end{thebibliography}

\section{Appendix A1}
\subsection{Series System Failure Rate}

\textbf{Definitions:} \\
\noindent
Reliability: $R(t) = 1 - F(t)$ \\
Hazard or failure rate function: $\lambda(t) = \frac{f(t)}{R(t)}$ \\
Series system reliability: $R_S(t) = \prod_{i = 1}^N R_i(t)$
\\
\\
\textbf{Result:}
$$
\begin{aligned}
	R_S(t) &= \prod_{i = 1}^N R_i(t) \\
	\frac{dR_S(t)}{dt} &= \frac{d}{dt} \prod_{i = 1}^N R_i(t) \; \; \text{(derivative of both sides)} \\
	\frac{dR_S(t)}{dt} &= \sum_{i=1}^N \left[ \frac{dR_i(t)}{dt} \frac{dR_S(t)}{dR_i(t)} \right] \; \; \text{(product rule)} \\
    \frac{-\frac{dR_S(t)}{dt}}{R_S(t)} &= \sum_{i=1}^N \left[  \frac{-\frac{d}{dt}R_i(t)}{dR_i(t)} \right] \; \; \text{(divide both sides by $-R_S(t)$)} \\
    \lambda_S(t) &= \sum_{i = 1}^N \lambda_i(t) \; \; \text{(because $\lambda(t) = \frac{f(t)}{R(t)} = \frac{-\frac{dR(t)}{dt}}{R(t)}$ )}
\end{aligned}
$$

\subsection{Non-Homogeneous Poisson Process}

\textbf{Definitions:} \\
\noindent
Number of failures before time t: $N(t) \sim \text{Poisson}(m(t))$ \\
Mean function: $m(t) = \int_0^t \lambda(s)ds$ (represents the expected number of failures before time t) \\
Weibull($\gamma, \beta $) failure rate: $\lambda(t) = \gamma\beta t^{\beta - 1}$
\\
\\
\textbf{Result:}\\
If we assume a constant shape parameter $\beta$ then,\\
$$
\begin{aligned}
	\lambda_S(t) &= \sum_{i = 1}^N \lambda_i(t) \\
    &= \sum_{i = 1}^N \gamma_i\beta t^{\beta - 1} \\
    &= \beta t^{\beta -1} \sum_{i = 1}^N \gamma_i
\end{aligned}
$$
\newpage
Then Solving for the mean function of the system,

$$
\begin{aligned}
	m_S(t) &= \int_0^t \lambda_S(s)ds \\
    &= \int_0^t \beta t^{\beta -1} \sum_{i = 1}^N \gamma_i \\
    &= \sum_{i = 1}^N \gamma_i \int_0^t \beta t^{\beta -1} \\
    &= t^\beta \sum_{i = 1}^N \gamma_i
\end{aligned}
$$

This gives us: $N(t) \sim \text{Poisson}(t^\beta \sum_{i = 1}^N \gamma_i)$

\end{document}
