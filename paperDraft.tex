\documentclass[12pt]{article}
\usepackage{amsmath}
\usepackage{graphicx,psfrag,epsf}
\usepackage{enumerate}

\begin{document}

\title{Bayesian Hierarchical Models for Common Components across Multiple System Configurations}

\begin{abstract}
    We use a Bayesian hierarchical model to assess the reliability of the Joint Light Tactical
    Vehicle (JLTV), which is a family of vehicles. The proposed model effectively combines information
    across three phases of testing and across common vehicle components. The analysis yields
    estimates of failure rates for specific failure modes and vehicles as well as an overall estimate of the
    failure rate for the family of vehicles. We are also able to obtain estimates of how well vehicle
    modifications between test phases improve failure rates. In addition to using all data to improve on
    current assessments of reliability and reliability growth, we illustrate how to leverage the information
    learned from the three phases to determine appropriate specifications for subsequent testing that will
    demonstrate if the reliability meets a given reliability threshold.
\end{abstract}


\section{Introduction}

\section{Data}


\section{Methodology}

\subsection{Modeling Reliability}
A standard reliability analysis employed by the Department of Defense (DoD) test
community considers each test phase independently and uses the exponential
distribution to model the miles between failure ~\cite{ref1}. The traditional
analysis is overly simplistic, relies on correct modeling assumptions
and ignores valuable information learned about the individual vehicles and their
failure modes.

In considering the following alternative approach we will begin by introducing a
hierarchical model structure that can use data across all test phases and
incorporate known similarities between vehicle failure modes.  Next, we will
explore the distributional assumptions of the model; how we can test these
assumptions and how misspecifications can effect our assessments of system
reliability.  This will then lead us into the next section on how this modeling
can be used for assurance test planning.

\subsubsection{Exponential Model}
In Test Phase 1, we assume the vehicle miles at failure, $y$, follow an
exponential distribution with a failure rate parameter $\lambda_{ij}$.
Introducing notation,
\begin{align}
y_{ijk}\mid\lambda_{ij}\sim Exp(\lambda_{ij}), \quad i = 1,2,..,v \quad j=1,2,..,s \quad k=1,2,..,n_{ij}
\end{align}
where $y_{ijk}$ are the miles between failure for vehicle $i$ failure mode $j$,
$v$ is the number of vehicles, $s$ is the number of failure modes, and $n_{ij}$
are the number of failures of vehicle $i$ failure mode $j$. The number of
failure modes is assumed fixed and known \textit{a priori}.

The prior distribution on the exponential failure rate parameter,
$\lambda_{ij}$, depends on whether failure mode $j$ is considered to be common
across vehicles or related but not identical. For the related failure modes,
we place a gamma prior distribution on the collection of $\lambda_{ij}$; in
other words, we assume each vehicle has a distinct failure rate in failure mode
$j$ but they arise from a common gamma distribution.

If failure mode $j$ is considered common across vehicles, the collection of
failure rates is collapsed to a single parameter, $\lambda_{ij} = \lambda_j$. As
with the related failure modes, a gamma prior distribution is placed on the
single failure rate. The prior distributions are independent across failure
mode, and can potentially have different hyperparameters.

The Phase 1 analysis yields an estimate of the failure rate $\lambda_{ij}$ for
each of the vehicles for failure modes that are related. We are assuming the
vehicles are conditionally independent, therefore the failure rate estimate for
the family of vehicles for such failure modes can be found by
$\sum_{i}\lambda_{ij}$. For failure modes that are common across vehicles the
Phase 1 analysis yields a $\lambda_{j}$, which is the failure rate for the
family of vehicles. Under the exponential modeling assumption the overall
failure rate across all failure modes can be found by $\sum_{j}\lambda_{j}$.

\subsubsection{Fix Effectiveness}
After the first CAP, Test Phase 2 begins with the repaired vehicles. To capture
these revisions, the PM2 reliability growth model \cite{EH06} is often used.
This model explicitly captures testing phases, choices about which failure modes
to correct, and the potential of not completely eliminating a failure upon
repair. One of the downsides of PM2 is that many parameters of potential
interest, such as the Fix Effectiveness Factor (FEF), which measures how much
repairs improve failure rates, are typically fixed. A common value for FEF is
0.70. We follow the premise of this type of model, but allow a more flexible and
data-driven result that is less dependent on hard-coded assumptions.

One normal assumption used in reliability growth modeling is non-decreasing
failure rates; that is either the fixes were effective or had no effect, but did
not degrade the family of vehicles.  This should generally be the case, but
because we are dealing with complex systems we will sometimes see decreases in
failure rates after adjustments are made.  Therefore for the Phase 2 data we
write the rate parameters as a function of the rate parameters found in Phase 1.
In particular, we define $\lambda_{ij}^{P2}=(\rho_{j})\lambda_{ij}^{P1}$ where
$\rho_{j}$ represents the between phase change in failure mode $j$. Given this
definition of $\lambda_{ij}^{P2}$, we again model the failure miles for a given
vehicle and failure mode using the exponential distribution. We assume the prior
distribution for the $\rho_{j}$ is a gamma distribution.  If $\rho_{j}$ is less
than one, this represents an improvement in reliability.  After Phase 2 we can
look again at failure rates across failure modes and vehicles and obtain an
overall estimate of the rate for the family of vehicles.  The analysis of Test
Phase 3 follows the same pattern as that shown in Phase 2. At the end of
Phase 3, we can look at failure rates across failure modes and vehicles and
obtain an overall estimate of the rate for the family of vehicles.
Future tests will be planned based on the inferences of Phase 3.

\subsubsection{Weibull Model}
The exponential model is by far the most common parametric distribution used in
reliability modeling because of its desirable math mathematical properties and
simple interpretations.   Dispite its common uses the assumption of a constant
failure rate over time is rarely justifiable.  It has been well documented
(Statistics, Testing, and Defense Acquisition: Background Papers chapter - 2 http://www.nap.edu/catalog/9655.html)
the issues that can arise when this assumption is violated.  We will now
consider the same hierarchical model structure while using the Weibull
distribution for each miles between failure observation.
\begin{align}
y_{ijk}\mid\lambda_{ij}\kappa_{i}\sim Weibull(\lambda_{ij}), \quad i = 1,2,..,v \quad j=1,2,..,s \quad k=1,2,..,n_{ij}
\end{align}
The Weibull distribution is a more flexible model with both a rate parameter
$\lambda_{ij}$ and a shape parameter $\kappa_{i}$.  The exponential is a special
case of the Weibull, when $\kappa = 1$.
\\
Talk about how to index the shape parameter and what it means

\subsubsection{JLTV Model Fitting}
So far we have discussed a number of different models and the assumptions that
accompany them.  Now looking at the JLTV dataset we will run a few diagnostics to
decide what model we will use for inference and future test planning.
\\
Models: \\
Exponential and Weibull with no hierarchy \\
Exponential and Weibull full hierarchy \\
Exponential and Weibull full hierarchy \\
Weibull with multipule shape vs. one shape \\
\\
DIC \\
posterior predictive checks \\
QQ Plots \\
Shape Parameter Plot \\

\subsubsection{JLTV Reliability Results}

Results plots \\
Exponential vs. Weibull interpretation

\subsection{Assurance Test Planning}
Assurance Testing
consumer and producer risk

\subsubsection{Traditional Approach}
no producer risk

\subsubsection{Bayes Risk}

R(t) = reliability at time t (miles in our case)
\\
\\$ t_{*c} $ : time of interest to consumer
\\$ t_{*p} $ : time of interest to producer
\\
\\
\begin{itemize}
\item Consumer Risk :   Prob( $ R(t_{*c}) \leq \pi_c \vert $ Test is passed)
\item Producer Risk :   Prob( $ R(t_{*p}) \geq \pi_p \vert $ Test is failed)
\end{itemize}
\
\\$ \pi_c $ : minimum reliability acceptable to the consumer at $ t_{*c} $
\\$ \pi_p $ : minimum reliability goal of the producer at $ t_{*p} $
\\
\\
We would like to get reliability into a inequality in terms of something we have
a distribution for so we can evaluate the conditional probability statements.
\\
\\
\\$ R(t_{*c}) \leq \pi_c \Rightarrow $ average number of system failures in 80
miles is greater than 2
\\$ R(t_{*p}) \geq \pi_p \Rightarrow $ average number of system failures in 140
miles is less than 2

\subsubsection{Poisson Process}

Reliability: $R(t) = 1 - F(t)$ \\
Hazard or failure rate function: $\lambda(t) = \frac{f(t)}{R(t)}$ \\
Series system reliability: $R_S(t) = \prod_{i = 1}^N R_i(t)$
\\
\\
\textbf{Result:}
$$
\begin{aligned}
	R_S(t) &= \prod_{i = 1}^N R_i(t) \\
	\frac{dR_S(t)}{dt} &= \frac{d}{dt} \prod_{i = 1}^N R_i(t) \; \; \text{(derivative of both sides)} \\
	\frac{dR_S(t)}{dt} &= \sum_{i=1}^N \left[ \frac{dR_i(t)}{dt} \frac{dR_S(t)}{dR_i(t)} \right] \; \; \text{(product rule)} \\
    \frac{-\frac{dR_S(t)}{dt}}{R_S(t)} &= \sum_{i=1}^N \left[  \frac{-\frac{d}{dt}R_i(t)}{dR_i(t)} \right] \; \; \text{(divide both sides by $-R_S(t)$)} \\
    \lambda_S(t) &= \sum_{i = 1}^N \lambda_i(t) \; \; \text{(because $\lambda(t) = \frac{f(t)}{R(t)} = \frac{-\frac{dR(t)}{dt}}{R(t)}$ )}
\end{aligned}
$$

\subsubsection{Non-Homogeneous Poisson Process}

\textbf{Definitions:} \\
\noindent
Number of failures before time t: $N(t) \sim \text{Poisson}(m(t))$ \\
Mean function: $m(t) = \int_0^t \lambda(s)ds$ (represents the expected number of failures before time t) \\
Weibull($\gamma, \beta $) failure rate: $\lambda(t) = \gamma\beta t^{\beta - 1}$
\\
\\
\textbf{Result:}\\
If we assume a constant shape parameter $\beta$ then,\\
$$
\begin{aligned}
	\lambda_S(t) &= \sum_{i = 1}^N \lambda_i(t) \\
    &= \sum_{i = 1}^N \gamma_i\beta t^{\beta - 1} \\
    &= \beta t^{\beta -1} \sum_{i = 1}^N \gamma_i
\end{aligned}
$$
\newpage
Then Solving for the mean function of the system,

$$
\begin{aligned}
	m_S(t) &= \int_0^t \lambda_S(s)ds \\
    &= \int_0^t \beta t^{\beta -1} \sum_{i = 1}^N \gamma_i \\
    &= \sum_{i = 1}^N \gamma_i \int_0^t \beta t^{\beta -1} \\
    &= t^\beta \sum_{i = 1}^N \gamma_i
\end{aligned}
$$

This gives us: $N(t) \sim \text{Poisson}(t^\beta \sum_{i = 1}^N \gamma_i)$

\subsubsection{Exponential Case}

If we model the reliability of all 26 components in the system as $Y_{ij} \sim$
exponential$\;(\gamma_i) $ random variables, this leads to the system
reliability being the minimum or $Y_{system} \sim $ exponential
$ \;(\sum_{n = 1}^{26} \gamma_i) $
\\
\\
- Looking at consumer risk first:
\\
want to find Prob( $ R(t_{*c}) \leq \pi_c \vert $ Test is passed)
\\
Let $ \sum_{n = 1}^{26} \gamma_i = \lambda_S $
\\
The expected number of failures for the system per mile is
$ \mathbf{E}(Y_{system}) = \lambda_S $
\\
\\
This leads to our consumer risk probability constraint as follows.  Given the
test is passed, the consumer would like the probability of the expected number
of failures in 80 miles being greater than 2 to be smaller than $\alpha$.
\\
\\
Prob( $ \lambda_S \cdot (80) \geq 2 \; \vert $ Test is passed) $ \leq \alpha $
\\
\\
Let $ W $ be the number of failures during the test and $ W \leq c \Rightarrow $
Test is passed and $ W > c \Rightarrow $ Test is failed.
\\
\\
Because failures are exponential $ W \sim Poisson(\lambda_S T) $ where $ T $ is
the number of miles run during the test.
\\
\\
$$
\begin{aligned}
	 P(\lambda_S \geq 2/80 \; \vert \; W \leq c) &= \int_{1/40}^{\infty} P(\lambda_S \; \vert W < c) d\lambda_S \\
     &= \int_{1/40}^{\infty} \frac{f(W < c \; \vert \lambda_S) p(\lambda_S)}{f(W < c)}d\lambda_S\\
     &= \int_{1/40}^{\infty} \frac{f(W < c \; \vert \lambda_S) p(\lambda_S)}{\int_{0}^{\infty} f(W < c \; \vert \lambda_S) p(\lambda_S) d\lambda_S}d\lambda_S \\
     &= \frac{\int_{1/40}^{\infty} [ \sum_{W=0}^c \frac{(\lambda_S T)^W exp(-\lambda T)}{W!}]p(\lambda_S)d\lambda_S} {\int_{0}^{\infty} [ \sum_{W=0}^c \frac{(\lambda T)^W exp(-\lambda_S T)}{W!}]p(\lambda_S)d\lambda_S}
\end{aligned}
$$
\\
\\
For simplicity we fix c to be zero (The number of failures needed to pass the
test) and take N posterior draws $ \lambda_S^{(j)} $
$$
\begin{aligned}
	 P(\lambda_S \geq 1/40 \; \vert \; W = 0) &= \frac{\int_{1/40}^{\infty} exp(-\lambda_S T)p(\lambda_S)d\lambda_S} {\int_{0}^{\infty} exp(-\lambda_S T)p(\lambda_S)d\lambda_S} \\
     &\approx \frac{\sum_{j = 1}^{N} exp(-\lambda_S^{(j)} T)I(\lambda_S^{(j)} \geq \frac{1}{40})} {\sum_{j = 1}^{N} exp(-\lambda_S^{(j)} T)}
\end{aligned}
$$
\\
\\
Using the same technique for producer risk we get the following. The producer
would like, given the test is failed, the probability of the expected number of
failures in 140 miles being less than 2 to be smaller than $\beta$.
\\
\\
$$
\begin{aligned}
	 P(\lambda_S \leq 1/70 \; \vert \; W > 0) &= \frac{\int_{0}^{1/70} [1 - exp(-\lambda_S T)]p(\lambda_S)d\lambda_S} {\int_{0}^{\infty} [1 - exp(-\lambda_S T)]p(\lambda_S)d\lambda_S} \\
     &\approx  \frac{\sum_{j = 1}^{N} [1 - exp(-\lambda_S^{(j)} T)] I(\lambda_S^{(j)} \leq \frac{1}{70})} {\sum_{j = 1}^{N} [1 - exp(-\lambda_S^{(j)} T)]}
\end{aligned}
$$
\\
\\
By constraining these two probabilities to acceptable risk levels we can solve
for the smallest $ T $ that satisfies both.
\\
\\
Consumer Risk : $ P(\lambda_S \geq 2/80 \; \vert  W = 0) \leq \alpha $ \\
Producer Risk : $ P(\lambda_S \leq 2/140 \; \vert  W > 0) \leq \beta $

\subsubsection{Weibull Case}
Now if we model the reliability of all 26 components in the system as
$Y_{ij} \sim$ Weibull$\;(\gamma_i, \beta) $ random variables we are able to use
the the non-homogeneous result from part two to build our assurance test. This
follows the same process as the exponential test plan with all mile variables
adjusted by the $\beta$ exponent and the N posterior draws will use both
$\lambda_S^{(j)}$ and $\beta^{(j)}$.
\\
\\
Consumer risk:
\\
$$
	 P(80^\beta (\lambda_S) \geq 2 \; \vert \; W = 0) \approx \frac{\sum_{j = 1}^{N} exp(-\lambda_S^{(j)} T^{\beta^{(j)}})I(80^{\beta^{(j)}} (\lambda_S^{(j)}) \geq 2)} {\sum_{j = 1}^{N} exp(-\lambda_S^{(j)} T^{\beta^{(j)}})}
$$
\\
\\
Producer risk:
\\
$$
	 P(140^\beta (\lambda_S) \leq 2 \; \vert \; W > 0) \approx  \frac{\sum_{j = 1}^{N} [1 - exp(-\lambda_S^{(j)} T^{\beta^{(j)}})] I(140^{\beta^{(j)}} (\lambda_S^{(j)}) \leq 2)} {\sum_{j = 1}^{N} [1 - exp(-\lambda_S^{(j)} T^{\beta^{(j)}})]}
$$
\\
\\
\textbf{Important note:} For the exponential test plan, the producer and
consumer risks were in terms of expected number of failures in a certain number
of miles, say $t$.  Because the exponential has a constant hazard rate this can
be considered the expected number of failures for a given number of miles
regardless of how many miles have been driven prior.  On the other hand the
Weibull does not have a constant hazard rate.  For the Weibull test plan
presented here the producer and consumer risk statements are now in terms of
expected number of failures in the first $t$ miles driven.



\subsubsection{Results}



\end{document}
