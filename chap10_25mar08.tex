\documentclass {book}
\usepackage{amsmath, amssymb, fullpage, natbib}
\usepackage{array}

\begin{document}


\chapter{Assurance Testing}\label{ch10}
\chaptermark{Assurance Testing}

\begin{quote}
\textit{Planning for Bayesian assurance testing involves
determining a test plan that guarantees that a reliability-related
quantity of interest meets or exceeds a specified requirement at a
desired level of confidence. Within a Bayesian hierarchical
framework, this chapter determines test plans for binomial, Poisson, and
Weibull testing. Also, we develop Weibull assurance test plans
using available data from an associated accelerated life testing
program.}
\end{quote} \index[sub]{assurance testing}\index[sub]{test!assurance}

\section{Introduction}\label{ch10:sec1}
This chapter focuses on developing a test plan \index[sub]{test
plan} for assuring (or demonstrating) that, at a desired level of
confidence, a reliability-related quantity of interest meets or
exceeds a specified requirement. For binomial testing, we test $n$
devices either as a demand for successful operation or for a
specified length of time and observe the total number of devices
failing the test $x$. For example, a tester may try an emergency
diesel generator (EDG) to see if it will start on demand, or a
tester may place a sample of a particular nonwoven material under
stress for a given length of time to see if it survives the test.
The reliability-related quantity of interest for both examples is
the probability $\pi$ that an item survives the test, and the
required binomial test plan\index[sub]{binomial!test plan}
consists of the total number of devices tested $n$ as well as the
maximum allowed number of failures $c$.

Although practitioners often use ``assuring'' and
``demonstrating'' synonymously, \citet{ME04}  distinguishes
between reliability demonstration and reliability assurance
testing. A traditional \emph{reliability demonstration
test}\index[sub]{reliability!demonstration test} is essentially a
\emph{classical} (i.e., frequency based) hypothesis test, which
uses only the data from the test to assess whether the
reliability-related quantity of interest meets or exceeds the
requirement. Consider how many modern systems, such as
communication devices and transportation systems, are highly
reliable. For these systems, reliability demonstration tests often
require an impractical amount of testing. In response to this
dilemma, \citet{ME04} defines an alternative \emph{reliability
assurance test}\index[sub]{reliability!assurance test}
\index[sub]{reliability!testing} as one that uses additional
supplementary data and information to reduce the required amount
of testing. The additional data and information may include
appropriate reliability models, earlier test results on the same
or similar devices, expert judgment regarding performance,
knowledge of the environmental conditions under which the devices
are used, benchmark design information on similar devices, prior
knowledge of possible failure modes, etc. Because all of the
Bayesian test plans considered in this chapter use such
supplementary data and information, we refer to them as
reliability assurance tests.\index[aut]{Meeker,
W.}\index[aut]{Escobar, L.}\index[aut]{Meeker,
W.}\index[aut]{Escobar, L.}

Life testing \index[sub]{test!life} has many aspects in common
with assurance testing. However, the primary goal in designing a
life test tends to be quite different than assuring conformance to
a specified reliability requirement. In designing such life tests,
we often have as our primary goal improving the estimation
precision of certain reliability-related quantities of interest.
However, such differences notwithstanding, the basic ideas
underlying life and assurance testing are similar, namely, to
address such questions as ``How many devices do I need to test?'',
``How long do I need to test each device?'', or ``What is the
maximum number of failures permitted for a successful test?''

Because data from an assumed sampling distribution provide the
basis for deciding whether the population of products being tested
meets the specified requirement, there are two kinds of errors to
make. A population of unreliable products (one that does not meet
the requirement) may, in fact, pass the test, whereas a reliable
population may fail it. This important acknowledgment makes us
think about the (probabilistic) \emph{risks} that we incur in
conducting the test. The precise form of the risks is an important
consideration in classical assurance tests and is an important
consideration in developing Bayesian assurance tests as well. The
\emph{test criteria}\index[sub]{test criteria} are precise
probabilistic statements regarding the risks we are willing to
incur when developing a test plan. The following sections discuss
several of the more popular criteria.

To begin our discussion of test criteria, suppose that $\pi$
denotes some reliability-related quantity of interest such that
large values of $\pi$ are more desirable than small values. Note
that reliability is one such quantity, while the mean and
quantiles of a specified failure time distribution are others. It
is common to base both classical and Bayesian test plans on two
specified levels of $\pi$: $\pi_0$, an \emph{acceptable
reliability level}\index[sub]{acceptable reliability
level}\index[sub]{ARL} (ARL), and $\pi_1$, a \emph{rejectable
reliability level}\index[sub]{rejectable reliability
level}\index[sub]{RRL} (RRL), where $\pi_1 \leq  \pi_0$. The
literature sometimes refers to the region $\pi_1 \leq \pi \leq
\pi_0$ as the \emph{indifference region}\index[sub]{indifference
region}. Although the precise definition of ARL and RRL differ
between the classical and Bayesian test criteria, we use them in
an equivalent way.

\subsection{Classical Risk Criteria}\label{ch10:sec1:ss1}
It is quite common to use two criteria in determining classical
test plans. The \emph{producer's risk}
\index[sub]{risk!producer's} is the probability of failing the
test when $\pi = \pi_0$, whereas the \emph{consumer's risk}
\index[sub]{risk!consumer's} is the probability of passing the
test when $\pi = \pi_1$.
\index[sub]{risk!producer's}\index[sub]{risk!consumer's} Suppose
that we specify a maximum value, $\alpha$, of the producer's risk
and a maximum value, $\beta$, of the consumer's risk. For binomial
testing, these criteria become
\begin{eqnarray}\label{ch10:eqn1}
Producer's\mbox{ }Risk & = & \PP (Test\mbox{ }Is\mbox{ }Failed
\con \pi_{0})\\
& = & \PP (y>c \con\pi_0)\nonumber\\
& = &
\sum_{y=c+1}^{n}\left(_y^n\right)(1-\pi_{0})^{y}\pi_{0}^{n-y} \leq
\alpha \, , \nonumber
\end{eqnarray}
and
\begin{eqnarray}\label{ch10:eqn2}
Consumer's\mbox{ }Risk & = & \PP( Test\mbox{ }Is\mbox{
}Passed \con \pi_1)\\
& = & \PP(y \leq c \con\pi_1) \nonumber\\
& = & \sum_{y=0}^{c} \left(_y^n\right)(1-\pi_{1})^
{y}\pi_{1}^{n-y} \leq \beta \, , \nonumber
\end{eqnarray}
where $\pi_1 \leq \pi_0$, $n$ is the number of test units, and $c$
is the maximum number of failures allowed.

To choose a test plan for specified values of $(\alpha, \pi_0,
\beta, \pi_1)$, we find the required binomial test plan $(n,c)$ by
simultaneously solving Eqs.~\ref{ch10:eqn1} and \ref{ch10:eqn2}.
Numerous textbooks provide additional details of this purely
classical approach, for example, see \citet{TT95}.
\index[aut]{Tobias, P.}\index[aut]{Trindade, D.}

\subsection{Average Risk
Criteria}\label{ch10:sec1:ss2}\index[sub]{risk!average}
\citet{E70} first proposed using average operating characteristics
and corresponding risk criteria. These risk criteria are similar
to the classical criteria in Sect.~\ref{ch10:sec1:ss1}, except
that now we condition on the events $\pi\geq\pi_0$ and
$\pi\leq\pi_1$, respectively. To do this requires a suitable prior
distribution for $\pi$,  as specified by $p(\pi)$. The
\emph{average producer's risk}\index[sub]{risk!average producer's}
is the probability of failing the test when $\pi\geq\pi_0$.
Choosing a maximum allowable average producer's risk $\alpha$, the
binomial test plan $(n,c)$ is \index[aut]{Easterling, R.}
\begin{eqnarray}\label{ch10:eqn3}
Average\mbox{ }Producer's\mbox{ }Risk & = &
\PP(Test\mbox{ }Is\mbox{ }Failed \con\pi \geq \pi_0)\\
& = & \frac {\PP(y>c,\pi \geq \pi_{0})}{\PP(\pi \geq \pi_{0})}\nonumber\\
& = & \frac{\int^{1}_{\pi_{0}} \left[ \sum^{n}_{y=c+1}
 \left(_y^n\right)(1-\pi)^{y}\pi^{n-y}\right]
p(\pi) \D \pi}{\int^{1}_{\pi_{0}}p(\pi) \D \pi}\nonumber\\
& = & \frac{\int^{1}_{\pi_{0}} \left[1 -\sum^{c}_{y=0}
\left(_y^n\right)(1-\pi)^{y}\pi^{n-y}\right]p(\pi) \D \pi}{
\int^{1}_{\pi_{0}}p(\pi)\D \pi} \leq \alpha \, .\nonumber
\end{eqnarray}
Likewise, the corresponding \emph{average consumer's
risk}\index[sub]{risk!average consumer's} is the probability of
passing the test when $\pi\leq\pi_1$. Choosing a maximum allowable
average consumer's risk $\beta$, the binomial test plan $(n,c)$ is
\begin{eqnarray}\label{ch10:eqn4}
Average\mbox{ }Consumer's\mbox{ }Risk & = &
\PP(Test\mbox{ }Is\mbox{ }Passed \con \pi \leq \pi_{1})\\
& = & \frac{\PP(y \leq c,\pi \leq \pi_{1})}{\PP(\pi \leq
\pi_1)}\nonumber\\
& = & \frac{\int^{\pi_1}_{0} \left[\sum^{c}_{y=0}
\left(_y^n\right)(1-\pi)^{y}\pi^{n-y}\right] p(\pi) \D
\pi}{\int^{\pi_1}_{0}p(\pi) \D \pi} \leq \beta \, .\nonumber
\end{eqnarray}
\citet{MW82} discusses the use of these risks and recommends
taking care in applying these criteria. For example, the average
consumer's risk may be a poor indication of what is likely really
desired; namely, a maximum probability $\beta$ that $\pi\leq\pi_1$
for a test that passes. The average consumer's risk given in
Eq.~\ref{ch10:eqn4} may be substantially larger than this desired
maximum conditional probability $\beta$. The prior probability
that $\pi \leq \pi_1$ may be quite small; however, if indeed $\pi
\leq \pi_1$, the probability of passing the test may be large. In
such a situation, using the average consumer's risk may be
inappropriate, and therefore misleading. \index[aut]{Martz, H.}
\index[aut]{Waller, R.}

\subsection{Posterior Risk
Criteria}\label{ch10:sec1:ss3}\index[sub]{risk!posterior} We now
consider fully Bayesian posterior risks that convey a completely
different outlook from the corresponding classical or average
risks. While the classical or average
risks\index[sub]{risk!average} provide assurance that satisfactory
devices will pass the test and that unsatisfactory devices will
fail it, posterior risks provide precisely the assurance that
practitioners often desire: if the test is passed, then the
consumer desires a maximum probability $\beta$ that $\pi \leq
\pi_1$. On the other hand, if the test is failed, then the
producer desires a maximum probability $\alpha$ that $\pi \geq
\pi_0$. Unlike the average risks, these posterior risks are fully
Bayesian in the sense that they are subjective probability
statements about $\pi$.

For a test that fails, the \emph{posterior producer's
risk}\index[sub]{risk!posterior producer's} is the probability
that $\pi\geq\pi_{0}$, or $\PP(\pi \ge \pi_0 \con Test\mbox{
}Is\mbox{ }Failed)$. Notice that this is simply the posterior
probability that $\pi \ge \pi_0$ given that we have observed more
than $c$ failures. Using Bayes' Theorem, and assuming a maximum
allowable posterior producer's risk $\alpha$, an expression for
the posterior producer's risk\index[sub]{risk!posterior
producer's} for the binomial test plan $(n,c)$\index[sub]{test
plan!binomial} is
\begin{eqnarray}\label{ch10:eqn5}
Posterior\mbox{ }Producer's\mbox{ }Risk & = &
\PP(\pi \geq \pi_{0}\con Test\mbox{ }Is\mbox{ }Failed)\\
\mbox{} & = & \int_{\pi_0}^{1} p(\pi \con y > c) \D \pi \nonumber\\
\mbox{} & = & \int_{\pi_0}^{1} \frac{f(y > c \con \pi)
p(\pi)}{\int_{0}^{1} f(y > c \con \pi) p(\pi) \D \pi} \,\, \D \pi \nonumber\\
\mbox{} & = & \frac{\int_{\pi_0}^{1} \left[
\sum^{n}_{y=c+1}(_y^n)(1-\pi)^{y}\pi^{n-y} \right] p(\pi) \D
\pi}{\int_{0}^{1} \left[
\sum^{n}_{y=c+1}(_y^n)(1-\pi)^{y}\pi^{n-y} \right] p(\pi) \D
\pi}\nonumber\\ \mbox{} & = & \frac{\int_{\pi_0}^{1} \left[
1-\sum^{c}_{y=0}(_y^n)(1-\pi)^{y}\pi^{n-y} \right] p(\pi) \D
\pi}{1-\int_{0}^{1} \left[\sum^{c}_{y=0}(_y^n)(1-\pi)^{y}\pi^{n-y}
\right] p(\pi) \D \pi} \leq \alpha.\nonumber
\end{eqnarray}

Similarly, given that the test is passed, the \emph{posterior
consumer's risk}\index[sub]{risk!posterior consumer's} is the
probability that $\pi \leq \pi_1$, or $\PP(\pi \leq \pi_1 \con
Test\mbox{ }Is\mbox{ }Passed)$. Notice that this is simply the
posterior probability that $\pi \leq \pi_1$ given that we have
observed no more than $c$ failures. Using Bayes' Theorem, and
assuming a maximum allowable posterior consumer's risk $\beta$, an
expression for the posterior consumer's
risk\index[sub]{risk!posterior consumer's} for the binomial test
plan\index[sub]{test plan!binomial} $(n,c)$ is
\begin{eqnarray}\label{ch10:eqn6}
Posterior\mbox{ }Consumer's\mbox{ }Risk & = &
\PP(\pi \leq \pi_{1}\con Test\mbox{ }Is\mbox{ }Passed)\\
\mbox{} & = & \int_{0}^{\pi_1} p(\pi \con y \leq c) \D \pi \nonumber\\
\mbox{} & = & \int_{0}^{\pi_1} \frac{f(y \leq c \con \pi)
p(\pi)}{\int_{0}^{1} f(y \leq c \con \pi) p(\pi) \D \pi} \,\, \D \pi \nonumber\\
\mbox{} & = & \frac{\int_{0}^{\pi_1} \left[
\sum^{c}_{y=0}(_y^n)(1-\pi)^{y}\pi^{n-y} \right] p(\pi) \D
\pi}{\int_{0}^{1} \left[ \sum^{c}_{y=0}(_y^n)(1-\pi)^{y}\pi^{n-y}
\right] p(\pi) \D \pi}\leq \beta.\nonumber
\end{eqnarray}

\begin{sidebar}\label{ch10:ex1} {\bf Binomial test plan for new modems.}
Consider finding a binomial test plan \index[sub]{test
plan!binomial} using the posterior consumer's risk criterion.
\citet{H90} develops a reliability assurance test for a new modem,
denoted by \emph{B}, that is similar to an earlier modem, denoted
by \emph{A}. Modem $A$ is currently in production and is very
reliable. The major difference between the two modems is that $B$
operates at a different frequency than $A$. Also, the same
production line that builds $A$ will produce $B$ and both modems
use most of the same components. Further, \citet{H90} reports that
a binomial assurance test for modem $A$ on 150 units yielded 6
failures.\index[aut]{Hart, L.}

One of the test objectives is to show that, after successful
testing, the 0.1 quantile of the posterior reliability
distribution for $B$ is at least 0.938, the 0.1 quantile of $A$'s
posterior reliability distribution. Similar to \citet{H90}, we use
a $Beta[86.4, 3.6]=Beta[(0.6 \times 150)(144/150), (0.6 \times
150)(6/150)]$ prior distribution for $\pi$. This prior
distribution arises from treating an $A$ test as ``worth'' 60\% of
a $B$ test or $90=0.6 \times 150$ total tests. Note that the 0.1
quantile of this prior distribution is 0.932, which is only
slightly smaller than the requirement. Therefore, we anticipate
that the test plan will require only a small sample of $B$ modems.
\index[aut]{Hart, L.}


A \emph{minimum sample size} (or \emph{zero-failure}) test plan is
one in which we test $n$ modems and state that the test is passed
if there are no failures, that is, $c = 0$. Given our
$Beta(86.4,3.6)$ prior distribution, $\pi_1 = 0.938$, $\beta =
0.10$, and $c = 0$, we find the desired Bayesian zero-failure test
plan by solving Eq.~\ref{ch10:eqn6} for sample size or number of
tests $n$. Using Eq.~\ref{ch10:eqn6} yields the expression
\index[sub]{test plan!minimum sample size}\index[sub]{test
plan!zero-failure}
\begin{eqnarray}\label{ch10:eqn7}
\lefteqn{\PP(\pi \leq 0.938 \con Test\mbox{ }Is\mbox{ }Passed)}\hspace{0.5in}\\
& & = \frac
{\int^{0.938}_{0}\left(_0^n\right)(1-\pi)^{0}\pi^{n}p(\pi) \D
\pi}{\int^{1}_{0} \left(_0^n\right)(1-\pi)^{0}\pi^{n}
p(\pi) \D \pi}\nonumber\\
& & = \frac {\int^{0.938}_{0}
\pi^{n}\frac{\Gamma(86.4+3.6)}{\Gamma(86.4)\Gamma(3.6)}
\pi^{86.4-1}(1-\pi)^{3.6-1} \D \pi}{\int^{1}_{0}
\pi^{n}\frac{\Gamma(86.4 + 3.6)}{\Gamma(86.4)\Gamma(3.6)}
\pi^{86.4-1}(1-\pi)^{3.6-1} \D \pi} \nonumber \\
& & = I(0.938;86.4 + n, 3.6) \leq 0.10 \, ,\nonumber
\end{eqnarray}
where $I(z; \alpha, \beta)$ denotes the \emph{incomplete beta
function ratio}\index[sub]{beta function!incomplete ratio}. Upon
evaluating the incomplete beta function ratio in
Eq.~\ref{ch10:eqn7} for increasing values of $n$, we find that $n
= 9$ is the smallest integer that satisfies the inequality.
Consequently, the plan consists of testing 9 $B$ modems. If none
fail, we can then claim that $\PP(\pi \leq 0.938 \con No\mbox{
}Failures\mbox{ }in\mbox{ }9\mbox{ }Tests) = 0.097 < 0.10$, as
required. In this case, given no failures in the 9 $B$ modem
tests, the 0.1 quantile of the posterior distribution of $\pi$ is
0.9384. Finally, the unconditional probability of passing the test
is simply
\begin{eqnarray*}
\PP[ Test\mbox{ }Is\mbox{ }Passed] & = & {\int^{1}_{0}
\left(_0^n\right)(1-\pi)^{0}\pi^{n}
p(\pi) \D \pi}\\
& =  & \frac{\Gamma(86.4+n)\Gamma(3.6)}{\Gamma(86.4+3.6+n)}
\frac{\Gamma(86.4+3.6)}{\Gamma(86.4)\Gamma(3.6)}\\
& = & \frac{\Gamma(95.4)\Gamma(90)}{\Gamma(99)\Gamma(86.4)} = 0.70. \\
\end{eqnarray*}
\end{sidebar}

\section{Binomial Testing}\label{ch10:sec2}\index[sub]{test
plan!binomial}\index[sub]{assurance
testing}\index[sub]{test!assurance} Now consider both the
average\index[sub]{risk!average} and posterior
risks\index[sub]{risk!posterior} for the binomial sampling
\index[sub]{sampling!binomial distribution} distribution within a
hierarchical framework. Suppose that we have failure count data
from $m > 1$ situations, such as $m$ different plants. Let $x_{i}$
denote the observed number of failures in a sample of size $n_{i}$
for $i = 1, \ldots, m$, and let $\vec{x}$ represent all the
observed failure count data. Then, conditional on the success
probability $\pi_{i}$, assume that the $X_i$ are conditionally
independent and that $X_{i} \con \pi_{i} \sim Binomial(n_{i}, 1 -
\pi_{i})$. Also assume that the $\pi_i$ can be modeled
hierarchically --- specifically, that given $\delta$ and $\gamma$
they are independent and identically distributed (i.i.d.)\ with a
common $Beta(\delta, \gamma)$ distribution. Finally, we specify a
prior distribution for the hyperparameters $(\delta, \gamma)$,
denoted by $p(\delta, \gamma)$, which is a proper (but usually
diffuse) joint distribution.


\subsection{Binomial Posterior Consumer's and Producer's
Risks}\label{ch10:sec2:ss1} Given that there are observed data
$X_i \sim Binomial(n_i, 1 - \pi_i)$, suppose that we are
interested in developing a binomial test plan $(n,c)$ using the
posterior risk\index[sub]{risk!posterior} criteria as our test
criteria. Our test plan is for a situation ``similar'' to those
previously observed, where we describe similarity by assuming that
for the new situation, an item has probability $\pi$ of surviving
the test, where given $\delta$ and $\gamma$, $\pi$ and the $\pi_i$
are i.i.d.\ $Beta(\delta, \gamma)$.

Recall that both the posterior producer's
risk\index[sub]{risk!posterior producer's} and posterior
consumer's risk\index[sub]{risk!posterior consumer's} specify
criteria on the posterior distribution for the binomial
probability of success $\pi$. Since there are now observed data
$\vec{x}$, we condition on that data and now use $p(\pi \con
\vec{x})$ in place of $p(\pi)$ in Eqs.~\ref{ch10:eqn5} and
\ref{ch10:eqn6}. In particular, for a binomial test plan $(n,c)$,
an expression for the posterior producer's risk is
\begin{eqnarray}\label{ch10:eqn8}
\PP(\pi \geq \pi_{0} & \con & Test\mbox{ }Is\mbox{ }Failed,\vec{x})\\
& =& \frac{\int_{\pi_0}^{1} \left[
1-\sum^{c}_{y=0}(_y^{n})(1-\pi)^{y}\pi^{n-y} \right] p(\pi \con
\vec{x}) \D \pi}{1-\int_{0}^{1}
\left[\sum^{c}_{y=0}(_y^{n})(1-\pi)^{y}\pi^{n-y} \right] p(\pi
\con \vec{x}) \D \pi}\, .\nonumber
\end{eqnarray}
Similarly, the posterior consumer's risk is
\begin{eqnarray}\label{ch10:eqn9}
\PP(\pi \leq \pi_{0} & \con & Test\mbox{ }Is\mbox{ }Passed,\vec{x})\\
& =& \frac{\int_{0}^{\pi_1} \left[
\sum^{c}_{y=0}(_y^{n})(1-\pi)^{y}\pi^{n-y} \right] p(\pi \con
\vec{x}) \D \pi}{\int_{0}^{1}
\left[\sum^{c}_{y=0}(_y^{n})(1-\pi)^{y}\pi^{n-y} \right] p(\pi
\con \vec{x}) \D \pi} \, .\nonumber
\end{eqnarray}

Now we must determine how to evaluate these criteria. There are
two equivalent ways to approach the problem. First, notice that
both of the posterior risk criteria are probability statements
about the posterior distribution of $\pi$ given different data.
For a given choice of $(n,c)$, the posterior producer's risk can
be calculated by using Markov chain Monte Carlo (MCMC) to find the
posterior distribution of $\pi$ given $\vec{x}$ and $y > c$ and
then calculating the proportion of posterior draws with $\pi \geq
\pi_0$. Similarly, for a given choice of $(n,c)$, the posterior
consumer's risk can be calculated by using MCMC to find the
posterior distribution of $\pi$ given $\vec{x}$ and $y \leq c$ and
then calculating the proportion of posterior draws with $\pi \leq
\pi_1$. Notice, however, that for each choice of $(n,c)$, this
requires using MCMC to calculate two posterior distributions for
$\pi$.

Suppose instead that we condition only the data $X_i \sim
Binomial(n_i, 1-\pi_i)$ with $\pi_i \sim Beta(\delta,\gamma)$ and
use MCMC to obtain the posterior predictive distribution $p(\pi
\con \vec{x})$. We can obtain draws from the posterior predictive
distribution $p(\pi \con \vec{x})$ using the $N$ posterior draws
for $(\delta, \gamma)$ by drawing $\pi^{(j)} \sim
Beta(\delta^{(j)}, \gamma^{(j)})$, and then use these samples to
evaluate Eqs.~\ref{ch10:eqn8} and \ref{ch10:eqn9} using Monte
Carlo integration. In general, to evaluate $\rE[g(x)] = \int g(x)
p(x) dx$, obtain a random sample $x_1, \ldots, x_N$ from $p(x)$
and approximate the expectation as $\frac{1}{N} \sum_{i=1}^{N}
g(x_i)$.

We evaluate the posterior producer's risk as
\begin{eqnarray*}
\lefteqn{\PP[\pi \geq \pi_{1}\con Test\mbox{ }Is\mbox{ }Failed,
\vec{x} ] \approx} \\
 & & \frac{\frac{1}{N} \sum^{N}_{j=1} \left[1-
\sum^{c}_{y=0}(_y^{n})(1-\pi^{(j)})^{y}(\pi^{(j)})^{n-y} \right]
I(\pi^{(j)} \geq \pi_0)}{1- \frac{1}{N}
\sum^{N}_{j=1}\left[\sum^{c}_{y=0}(_y^{n})(1-\pi^{(j)})^{y}(\pi^{(j)})^{n-y}
\right]} \, ,
\end{eqnarray*}
and the posterior consumer's risk as
\begin{eqnarray*}
\lefteqn{\PP[\pi \leq \pi_{1} \con Test\mbox{ }Is\mbox{ }Passed, \vec{x} ] \approx} \\
& & \frac{\sum^{N}_{j=1} \left[
\sum^{c}_{y=0}(_y^{n})(1-\pi^{(j)})^{y}(\pi^{(j)})^{n-y} \right]
I(\pi^{(j)} \leq \pi_1)}{\sum^{N}_{j=1}\left[
\sum^{c}_{y=0}(_y^{n})(1-\pi^{(j)})^{y}(\pi^{(j)})^{n-y} \right]}
\, .
\end{eqnarray*}
The expression for the unconditional probability of passing the
test is
\begin{equation}\label{ch10:eqn9a}
\PP(Test\mbox{ }Is\mbox{ }Passed \con \vec{x} ) \approx \frac
{1}{N}\sum^{N}_{j=1}\left[
\sum^{c}_{y=0}(_y^{n})(1-\pi^{(j)})^{y}(\pi^{(j)})^{n-y} \right]
\, .
\end{equation}
Let
\begin{displaymath}
b^{(j)}(y) = \frac{(_y^{n})B(n-y+\delta^{(j)},y+\gamma^{(j)})}
{B(\delta^{(j)},\gamma^{(j)})} \, ,
\end{displaymath}
where $B(\alpha, \beta)$ is the beta function. We can also
evaluate the posterior producer's risk using the posterior draws
$\delta^{(j)}, \gamma^{(j)} \con \vec{x}$ as
\begin{eqnarray*}
\lefteqn{\PP[\pi \geq \pi_{0}\con Test\mbox{ }Is\mbox{ }Failed,
\vec{x} ] \approx} \\
 & & \frac
{\sum^{N}_{j=1} \left[
1-I(\pi_{0};\delta^{(j)},\gamma^{(j)})-\sum^{c}_{y=0}
b^{(j)}(y)[1-I(\pi_{0};n-y+\delta^{(j)}, y+\gamma^{(j)})] \right]}
{\sum^{N}_{j=1} \left[1-\sum^{c}_{y=0} b^{(j)}(y) \right]} \, ,
\end{eqnarray*}
and the posterior consumer's risk as
\begin{eqnarray*}
\lefteqn{\PP[\pi \leq \pi_{1} \con Test\mbox{ }Is\mbox{ }Passed, \vec{x} ] \approx} \\
& & \frac {\sum^{N}_{j=1} \left[\sum^{c}_{y=0} b^{(j)}(y)
I(\pi_{1};n-y+\delta^{(j)},y+\gamma^{(j)})\right]}{\sum^{N}_{j=1}
\sum^{c}_{y=0} b^{(j)}(y)} \, ,
\end{eqnarray*}
where $I(z; \alpha, \beta)$ is the incomplete beta function ratio.
An additional expression for the unconditional probability of
passing the test is
\begin{eqnarray*}
\PP(Test\mbox{ }Is\mbox{ }Passed \con \vec{x} ) & \approx &
\frac{1}{N}\sum^{N}_{j=1} \sum^{c}_{y=0} b^{(j)}(y)\\
\mbox{} & \approx & \frac
{1}{N}\sum^{N}_{j=1}\left[\sum^{c}_{y=0}\frac{(_y^{n})B(n-y+\delta^{(j)},y+\gamma^{(j)})}
{B(\delta^{(j)},\gamma^{(j)})}\right] \, .
\end{eqnarray*}

To obtain a test plan, simultaneously solve the pair of
inequalities given by
\begin{equation}\label{ch10:eqn10}
\PP[\pi \geq \pi_{0}\con Test\mbox{ }Is\mbox{ }Failed, \vec{x} ]
\leq \alpha
\end{equation}
and
\begin{equation}\label{ch10:eqn11}
\PP[\pi \leq \pi_{1}\con Test\mbox{ }Is\mbox{ }Passed, \vec{x}
]\leq \beta,
\end{equation}
for the pair of integers $(n,c)$, where $0 \leq c < n$, and where
$\alpha$ and $\beta$ are the desired maximum posterior producer's
and consumer's risks.

We can find such test plans because Eqs.~\ref{ch10:eqn10} and
\ref{ch10:eqn11} have opposite effects. For fixed $c$, as $n$
increases, $\PP[\pi \leq \pi_{1} \con Test\mbox{ }Is\mbox{
}Passed, \vec{x} ]$ decreases, whereas $\PP[\pi \geq \pi_{0}\con
Test\mbox{ }Is\mbox{ }Failed, \vec{x} ]$ increases. On the other
hand, for fixed $n$, as $c$ increases, the opposite is true.
Consequently, we can use the algorithm in Fig.~\ref{ch10:fig1} to
find the required test plan.

\begin{figure}
\setlength{\unitlength}{0.095in} \centering
\begin{picture}(40,40)
\put(0,36){\framebox(5,4){Begin}} \put(2.5,36){\vector(0,-1){4}}
\put(0,28){\framebox(5,4){\shortstack{$n=1$ \\ $c=0$}}}
\put(2.5,28){\vector(0,-1){4}}
\put(0,20){\framebox(21,4){\shortstack{Posterior Producer's Risk
\\ $\PP[\pi \ge \pi_0 \con \mbox{Test is Failed}, \vec{x}] \le \alpha$?}}}
\put(2.5,20){\vector(0,-1){4}}
\put(0,12){\framebox(21,4){\shortstack{Posterior Consumer's Risk
\\ $\PP[\pi \le \pi_1 \con \mbox{Test is Passed}, \vec{x}] \le \beta$?}}}
\put(2.5,12){\vector(0,-1){4}} \put(0,4){\framebox(5,4){End}}
\put(24,20){\framebox(8,4){$c < n - 1$}}
\put(21,22){\vector(1,0){3}} \put(36,20){\framebox(8,4){$c = c +
1$}} \put(24,12){\framebox(8,4){$n = n + 1$}}
\put(36,12){\framebox(8,4){$n = n + 1$}}
\put(21,14){\vector(1,0){3}} \put(32,22){\vector(1,0){4}}
\put(16,18){\vector(0,1){2}} \put(16,18){\line(1,0){12}}
\put(28,18){\line(0,-1){2}} \put(30,20){\line(0,-1){2}}
\put(30,18){\line(1,0){10}} \put(40,18){\vector(0,-1){2}}
\put(40,24){\line(0,1){2}} \put(40,26){\line(-1,0){20}}
\put(20,26){\vector(0,-1){2}} \put(0,18){Yes} \put(0,10){Yes}
\put(22,22.5){No} \put(22,14.5){No} \put(33,22.5){Yes}
\put(33,18.5){No} \put(42,16){\vector(0,1){4}}
\end{picture}
\caption{An algorithm for finding Bayesian hierarchical binomial test
plans.}\label{ch10:fig1}
\end{figure}

\begin{sidebar}\label{ch10:ex2} {\bf Hierarchical binomial test plan for EDGs.}
\index[sub]{test plan!binomial} \citet{MKA96} analyzes failure
count data from EDGs in 63 U.S.\ commercial nuclear power plants
to assess their reliability of accepting the electrical load and
run (\emph{load-run}) on demand. See Table~\ref{ch10:tab1}, which
summarizes the data collected over the four-year period
1988-1991.\index[aut]{Martz, H.} \index[aut]{Kvam, P.}
\index[aut]{Abramson, L.}

\begin{table}
\caption{EDG load-run demand data ($x$ failures out of $n$
demands) \citep{MKA96}}\label{ch10:tab1} \centering
\begin{tabular}{crr|crr|ccr}
\hline \multicolumn{1}{c}{Plant} & \mbox{}  $x$ & \mbox{}  $n$
\mbox{} & \multicolumn{1}{c}\mbox{}
Plant & \mbox{}  $x$ & \mbox{}  $n$ \mbox{}&\multicolumn{1}{c}\mbox{}  Plant & $x$ & \mbox{}  $n$\\
\hline%
1  \mbox{}&  11 \mbox{} & 854\phantom{*} &\mbox{}  22  \mbox{} &  0  \mbox{} &   238 \mbox{}& \hspace{0.05in}43 &  5  \mbox{} &  216 \\
2  \mbox{}&  2  \mbox{} & 373\phantom{*} &\mbox{}  23  \mbox{} &  1  \mbox{} &   370 \mbox{}& \hspace{0.05in}44 &  0  \mbox{} &  252 \\
3  \mbox{}&  5  \mbox{} & 618\phantom{*} &\mbox{}  24  \mbox{} &  2  \mbox{} &   302 \mbox{}& \hspace{0.05in}45 &  1  \mbox{} &  419 \\
4  \mbox{}&  5  \mbox{} & 157\phantom{*} &\mbox{}  25  \mbox{} &  3  \mbox{} &   152 \mbox{}& \hspace{0.05in}46 &  0  \mbox{} &  136 \\
5  \mbox{}&  3  \mbox{} & 542\phantom{*} &\mbox{}  26  \mbox{} &  2  \mbox{} &   294 \mbox{}& \hspace{0.05in}47 &  3  \mbox{} &  185 \\
6  \mbox{}&  2  \mbox{} & 202\phantom{*} &\mbox{}  27  \mbox{} &  0  \mbox{} &   101 \mbox{}& \hspace{0.05in}48 &  7  \mbox{} &  382 \\
7  \mbox{}&  0  \mbox{} & 65\phantom{*}  &\mbox{}  28  \mbox{} &  0  \mbox{} &   283 \mbox{}& \hspace{0.05in}49 &  6  \mbox{} &  304 \\
8  \mbox{}&  2  \mbox{} & 166\phantom{*} &\mbox{}  29  \mbox{} &  2  \mbox{} &   117 \mbox{}& \hspace{0.05in}50 &  2  \mbox{} &  130 \\
9  \mbox{}&  4  \mbox{} & 574\phantom{*} &\mbox{}  30  \mbox{} &  0  \mbox{} &   115 \mbox{}& \hspace{0.05in}51 &  1  \mbox{} &  121 \\
10 \mbox{}&  2  \mbox{} & 201\phantom{*} &\mbox{}  31  \mbox{} &  1  \mbox{} &   196 \mbox{}& \hspace{0.05in}52 &  2  \mbox{} &  295 \\
11 \mbox{}&  8  \mbox{} & 388\phantom{*} &\mbox{}  32  \mbox{} &  0  \mbox{} &   310 \mbox{}& \hspace{0.05in}53 &  1  \mbox{} &  289 \\
12 \mbox{}&  2  \mbox{} & 287\phantom{*} &\mbox{}  33  \mbox{} &  4  \mbox{} &   134 \mbox{}& \hspace{0.05in}54 &  0  \mbox{} &  181 \\
13 \mbox{}&  3  \mbox{} & 431\phantom{*} &\mbox{}  34  \mbox{} &  2  \mbox{} &   132 \mbox{}& \hspace{0.05in}55 &  2  \mbox{} &  150 \\
14 \mbox{}&  5  \mbox{} & 358\phantom{*} &\mbox{}  35  \mbox{} &  0  \mbox{} &   242 \mbox{}& \hspace{0.05in}56 &  0  \mbox{} &  334 \\
15 \mbox{}& 14  \mbox{} & 1120\phantom{*} &\mbox{} 36  \mbox{} &  4  \mbox{} &   132 \mbox{}& \hspace{0.05in}57 &  2  \mbox{} &  263 \\
16 \mbox{}&  1  \mbox{} & 321\phantom{*} &\mbox{}  37  \mbox{} &  4  \mbox{} &   320 \mbox{}& \hspace{0.05in}58 &  0  \mbox{} &  92 \\
17 \mbox{}&  3  \mbox{} & 225\phantom{*} &\mbox{}  38  \mbox{} &  0  \mbox{} &   996 \mbox{}& \hspace{0.05in}59 &  2  \mbox{} &  466 \\
18 \mbox{}&  8  \mbox{} & 433\phantom{*} &\mbox{}  39  \mbox{} &  1  \mbox{} &   253 \mbox{}& \hspace{0.05in}60 &  2  \mbox{} &  387 \\
19 \mbox{}&  7  \mbox{} & 468\phantom{*} &\mbox{}  40  \mbox{} & 13  \mbox{} &   704 \mbox{}& \hspace{0.05in}61 &  1  \mbox{} &  183 \\
20 \mbox{}&  1  \mbox{} & 218\phantom{*} &\mbox{}  41  \mbox{} &  2  \mbox{} &   151 \mbox{}& \hspace{0.05in}62 &  5  \mbox{} &  278 \\
21 \mbox{}&  9  \mbox{} & 317\phantom{*} &\mbox{}  42  \mbox{} &  2  \mbox{} &   385 \mbox{}& \hspace{0.05in}63 &  0  \mbox{} &  212\\
\noalign{\smallskip}\hline
\end{tabular}
\end{table}

We model the failure counts $X_i$ as conditionally independent given
$\pi_i$ with $Binomial(n_i,1 - \pi_i)$ distributions, where
$\pi_i$ is the $i$th plant reliability. Because the plants have
all been built and operated to the same Nuclear Regulatory
Commission (NRC)-controlled safety standards, model the 63
$\pi_i \con \delta, \gamma$ as i.i.d.\ with a common
$Beta(\delta,\gamma)$ distribution. In our analysis of these data,
we use independent and diffuse $InverseGamma(0.1,0.1)$ prior
distributions for $\delta$ and $\gamma$. See
Table~\ref{ch10:tab2}, which summarizes the marginal posterior
distributions for $\delta$ and $\gamma$ and the predictive
distribution for $\pi$. The hierarchical binomial model fits the
load-run demand data well (see
Exercise~\ref{ch4}.\ref{exercise:ch4EDG}).

\begin{table}
\caption{Posterior distribution summaries for the hyperparameters
$\delta$ and $\gamma$ and the predictive distribution for
$\pi$}\label{ch10:tab2}
 \centering
\begin{tabular}{crrrrrrrrrrrrrr}
\hline                              &       &    &       &       &
 &                                & & \multicolumn{3}{c}{Quantiles}  & &
 \\ \cline{7-15}
 Parameter&\mbox{}&Mean&\mbox{}&Std
 Dev&\mbox{}\mbox{}\mbox{}
 &\multicolumn{1}{r}{0.025}&\mbox{}&\multicolumn{1}{r}{0.05}&\mbox{}&0.50 & \mbox{} & \multicolumn{1}{r}{0.95}& \mbox{} & \multicolumn{1}{r}{0.975}\\
\hline%
$\delta$ &\mbox{}   &286.4&\mbox{}  &135.6&\mbox{}  &121.5&\mbox{}  &136.9&\mbox{}  &253.2&\mbox{}  &557.4&\mbox{}   &652.8\\
$\gamma$&\mbox{}    &2.710&\mbox{}  &1.243&\mbox{}  &1.226&\mbox{}  &1.360&\mbox{}  &2.402&\mbox{}  &5.171&\mbox{}   &6.204\\
$\pi$&\mbox{}& 0.9903 &\mbox{}& 0.0067 &\mbox{}& 0.9735 &\mbox{}& 0.9778 &\mbox{}& 0.9917 &\mbox{}& 0.9981 &\mbox{}& 0.9987 \\
\hline
\end{tabular}
\end{table}

Using the posterior risk criteria, suppose now that we want to
find the test plan having the posterior consumer's and producer's
risks $\beta = 0.05$ for $\pi_{1} = 0.985$, and $\alpha = 0.10$
for $\pi_{0} = 0.999$, respectively.

Using the algorithm in Fig.~\ref{ch10:fig1}, we obtain the
required test plan $n = 42$ and $c = 0$. For this test plan, the
actual posterior consumer's risk \linebreak $\PP[\pi \leq 0.985
\con Test\mbox{ }Is\mbox{ }Passed, \vec{x}] = 0.0014$ and the
actual posterior producer's risk $\PP[\pi \geq 0.999\con
Test\mbox{ }Is\mbox{ }Failed,\vec{x}] = 0.0992$. Therefore, at the
new plant, load and run the EDGs 42 times. (Note that we are also
assuming here that all the EDGs at the new plant have the same
load-run demand reliability.) If there are no failures, then we
can assure an EDG load-run demand reliability of 0.985 with a 0.95
probability. On the other hand, if the test is failed, we have the
assurance that the EDG load-run demand reliability is not greater
than 0.999 with a 0.90 probability. Finally, from
Eq.~\ref{ch10:eqn9a}, the unconditional probability of passing
this test is approximately 0.69.
\end{sidebar}

\subsection{Hybrid Risk Criterion}\label{ch10:sec1:ss4}\index[sub]{risk!hybrid}
\citet{FMP99} uses the posterior consumer's
risk\index[sub]{risk!posterior consumer's} criterion in
Eq.~\ref{ch10:eqn6} and the average producer's risk criterion in
Eq.~\ref{ch10:eqn3} to \index[sub]{risk!average producer's}
determine alternative binomial test plans. These ``hybrid''
criteria are appropriate in situations where we are interested in
a single specified level of reliability, which \citet{FMP99}
refers to as the \emph{target reliability
level}\index[sub]{reliability!target}. \citet{FMP99} reasons that,
in many applications of reliability assurance testing, the
consumer and producer are one and the same. Consequently, in such
situations, there may be less adversarial tension in determining
appropriate risk criteria, and therefore, little motivation to
specify two different (and acceptable) levels of reliability
($\pi_{0}$ and $\pi_{1}$) for the required test
plan.\index[aut]{Fitzgerald, M.} \index[aut]{Martz, H.}
\index[aut]{Parker, R.}\index[aut]{Fitzgerald, M.}
\index[aut]{Martz, H.} \index[aut]{Parker, R.}

Likewise, we use this hybrid approach and seek a Bayesian binomial
test plan for which there is a probability of at most $\beta$ that
$\pi \leq \pi^{*}$ for a passing test, and simultaneously, there
is a probability of at most $\alpha$ that a test is failed given
$\pi \geq \pi^{*}$.

Notationally, we want to find the test plan $(n,c)$ that satisfies
the hybrid risk criteria\index[sub]{risk!hybrid} given by
\begin{equation}\label{ch10:eqn12}
\PP[Test\mbox{ }Is\mbox{ }Failed \con\pi\geq\pi^{*}, \vec{x}] \leq
\alpha
\end{equation}
and
\begin{equation}\label{ch10:eqn13}
\PP[\pi \leq \pi^{*} \con Test\mbox{ }Is\mbox{ }Passed,\vec{x}]
\leq \beta.
\end{equation}

For test plan $(n,c)$, we can approximate the average producer's
risk by
\begin{eqnarray}\label{ch10:eqn14}
\lefteqn{\PP[Test\mbox{ }Is\mbox{ }Failed \con \pi\geq\pi^{*},
\vec{x}] \approx} \\
& & \frac{\sum^{N}_{j=1} \left[
1-I(\pi^{*};\delta^{(j)},\gamma^{(j)})-\sum^{c}_{y=0} b^{(j)}(y)
[1-I(\pi^{*};n-y+\delta^{(j)}, y+\gamma^{(j)})] \right]}
{\sum^{N}_{j=1} \left[ 1-I(\pi^{*};\delta^{(j)},\gamma^{(j)})
\right] } .\nonumber
\end{eqnarray}

We find the desired Bayesian test plan by employing the algorithm
in Fig.~\ref{ch10:fig1}, where Eq.~\ref{ch10:eqn14} replaces the
producer's risk inequality statement used in the algorithm.

\begin{sidebar}\label{ch10:ex3} {\bf Hierarchical binomial test plan using hybrid
criteria.} Again consider Example~\ref{ch10:ex2}. Consider finding
a Bayesian test plan having the posterior consumer's and
average producer's risks $\beta = 0.05$ and $\alpha = 0.05$ for
$\pi^{*} = 0.98$, respectively. Using the algorithm in
Fig.~\ref{ch10:fig1} with the modification described above, we
obtain the required test plan $n = 71$ and $c = 2$. For this test
plan, the actual posterior consumer's risk is
$\PP[\pi\leq 0.98 \con Test\mbox{ }Is\mbox{ }Passed, \vec{x} ] =
0.0497$ and the actual average producer's risk $\PP[Test\mbox{
}Is\mbox{ }Failed \con \pi\geq 0.98, \vec{x}] = 0.0333$. Because we
now only consider a single target reliability level, this test
plan requires significantly more testing than that found in
Example~\ref{ch10:ex2}, in which two different reliability values
were specified. From Eq.~\ref{ch10:eqn9a}, the unconditional
probability of passing this test is approximately 0.951.
\end{sidebar}

\section{Poisson Testing}\label{ch10:sec3}\index[sub]{test
plan!Poisson}\index[sub]{assurance
testing}\index[sub]{test!assurance} We now move from binomial
testing within a hierarchical framework to Poisson testing. Recall
from Chap.~\ref{ch6} that a homogeneous Poisson
process\index[sub]{Poisson process!homogeneous} generates a
sequence of events for which the times between successive failures
(the \emph{interfailure times}) are independently and identically
$Exponential(\lambda)$ distributed. For a particular Poisson
testing situation, let us develop a Bayesian test plan for
assuring that the \emph{failure rate} $\lambda$ does not exceed a
specified requirement. From Poisson process theory, the number of
failures $X$ occurring in fixed total test or operating time $T$
has a $Poisson(\lambda T)$ distribution. Let $T$ represent the
total operating or exposure time of the devices during which we
assume that devices are either repaired or replaced when they
fail.

Suppose that the plan tests $n$ devices for the length of time
$t_{0}$, replacing devices as they fail, so that the total
operating time is $T = nt_{0}$. In determining a Bayesian test
plan, we seek $(T,c)$, where $c$ is the maximum allowed number of
failures. The test is passed if no more than $c$ failures occur in
total test time $T$. Note that any combination of $n$ and $t_{0}$
satisfying $T = nt_{0}$ provides an acceptable test plan.

As with binomial testing, let us assume in Poisson testing that
there are data available from $m > 1$ situations. Let $x_{i}$
denote the observed number of failures in total operating time
$T_{i}$ for the $i$th situation, and let $\vec{x}$ represent all
the observed failure data. Then, conditioning on $\lambda_{i}$,
$X_{i} \sim Poisson(\lambda_{i}T_{i})$, where the $X_{i}$ are
conditionally independent. We model the $\lambda_{i}$
hierarchically, assuming they are i.i.d.\ $Gamma(\eta, \kappa)$,
given $\eta$ and $\kappa$, and specify a prior distribution for
the hyperparameters $(\eta, \kappa)$, denoted by $p(\eta,
\kappa)$.

We can write an expression for the posterior producer's
risk\index[sub]{risk!posterior producer's} for the Poisson test
plan\index[sub]{test plan!Poisson} $(T,c)$, where $\lambda_{0}$ is
the rejectable failure rate and there is a maximum allowable
posterior producer's risk $\alpha$:
\begin{eqnarray}\label{ch10:ppr}
Posterior\mbox{ }Producer's\mbox{ }Risk & = &
\PP(\lambda \leq \lambda_{0}\con Test\mbox{ }Is\mbox{ }Failed)\\
\mbox{} & = & \int_{0}^{\lambda_0} p(\lambda \con y > c) \D \lambda \nonumber\\
\mbox{} & = & \int_{0}^{\lambda_0} \frac{f(y > c \con \lambda)
p(\lambda)}{\int_{0}^{\infty} f(y > c \con \lambda) p(\lambda) \D \lambda} \,\, \D \lambda \nonumber\\
\mbox{} & = & \frac{\int_{0}^{\lambda_0} \left[1-
\sum^{c}_{y=0}\frac{(\lambda T)^y \exp(-\lambda T)}{y!} \right]
p(\lambda) \D \lambda}{\int_{0}^{\infty} \left[ 1- \sum^{c}_{y=0}
\frac{(\lambda T)^y \exp(-\lambda T)}{y!} \right] p(\lambda) \D
\lambda}\nonumber \leq \alpha.\nonumber
\end{eqnarray}

Similarly, given that the test is passed, we can write an
expression for the posterior consumer's
risk\index[sub]{risk!posterior consumer's} for the Poisson test
plan\index[sub]{test plan!Poisson} $(T,c)$, with acceptable
failure rate $\lambda_{1}$ and maximum allowable posterior
consumer's risk $\beta$, as
\begin{eqnarray}\label{ch10:pcr}
Posterior\mbox{ }Consumer's\mbox{ }Risk & = &
\PP(\lambda \geq \lambda_{1}\con Test\mbox{ }Is\mbox{ }Passed)\\
\mbox{} & = & \int_{\lambda_1}^{\infty} p(\lambda \con y \leq c) \D \lambda \nonumber\\
\mbox{} & = & \int_{\lambda_1}^{\infty} \frac{f(y \leq c \con
\lambda)
p(\lambda)}{\int_{0}^{\infty} f(y \leq c \con \lambda) p(\lambda) \D \lambda} \,\, \D \lambda \nonumber\\
\mbox{} & = & \frac{\int_{\lambda_1}^{\infty} \left[
\sum^{c}_{y=0}\frac{(\lambda T)^y \exp(-\lambda T)}{y!} \right]
p(\lambda) \D \lambda}{\int_{0}^{\infty} \left[\sum^{c}_{y=0}
\frac{(\lambda T)^y \exp(-\lambda T)}{y!} \right] p(\lambda) \D
\lambda}\nonumber \leq \beta.\nonumber
\end{eqnarray}

Since there are available data $\vec{x}$, we use the posterior
distribution for $\lambda$, $p(\lambda \con \vec{x})$, in
Eqs.~\ref{ch10:ppr} and \ref{ch10:pcr} to construct our test plan
for a new situation. We can calculate the posterior producer's
risk either using posterior predictive draws $\lambda^{(j)}$ or
using posterior draws $(\eta^{(j)}, \kappa^{(j)})$. Let
\begin{displaymath}
g^{(j)}(y) = \frac {(\kappa^{(j)})^{\eta^{(j)}}T^{y}}{y!
\Gamma(\eta^{(j)})(T+\kappa^{(j)})^{y+\eta^{(j)}}} \, .
\end{displaymath}

\begin{eqnarray*}
\PP[ \lambda & \leq & \lambda_{0} \con Test\mbox{ }Is\mbox{ }Failed,\vec{x}]\\
\mbox{} & \approx & \frac{\sum_{j=1}^{N} \left[1-
\sum^{c}_{y=0}\frac{(\lambda^{(j)} T)^y \exp(-\lambda^{(j)}
T)}{y!} \right]I(\lambda^{(j)} \leq \lambda_0)}{\sum_{j=1}^{N}
\left[ 1- \sum^{c}_{y=0}
\frac{(\lambda^{(j)} T)^y \exp(-\lambda^{(j)} T)}{y!} \right]}\\
\mbox{} & \approx & \frac { \sum^{N}_{j=1} \left[
\gamma(\eta^{(j)},\kappa^{(j)}\lambda_{0})/\Gamma(\eta^{(j)})-
\sum^{c}_{y=0} g^{(j)}(y)
\gamma[y+\eta^{(j)},(T+\kappa^{(j)})\lambda_{0}] \right]}
{\sum^{N}_{j=1} \left[ 1- \sum^{c}_{y=0} g^{(j)}(y)
\Gamma(y+\eta^{(j)})\right] } \, ,
\end{eqnarray*}
where $\gamma(q, z)$ denotes the \emph{lower incomplete gamma
function}\index[sub]{gamma function!lower incomplete}.

The expressions for the posterior consumer's
risk\index[sub]{risk!posterior consumer's} are
\begin{eqnarray*}
\PP[\lambda & \geq & \lambda_{1}  \con  Test\mbox{ }Is\mbox{
}Passed, \vec{x} ]\\
\mbox{} & \approx & \frac{\sum_{j=1}^{N} \left[
\sum^{c}_{y=0}\frac{(\lambda^{(j)} T)^y \exp(-\lambda^{(j)}
T)}{y!} \right] I(\lambda^{(j)} \geq \lambda_1)} {\sum_{j=1}^{N}
\left[\sum^{c}_{y=0}
\frac{(\lambda^{(j)} T)^y \exp(-\lambda^{(j)} T)}{y!} \right]}\\
& \approx & \frac { \sum^{N}_{j=1} \left[ \sum^{c}_{y=0}
g^{(j)}(y) \{ \Gamma(y+\eta^{(j)})
-\gamma[y+\eta^{(j)},(T_{0}+\kappa^{(j)})\lambda_{1}] \} \right] }
{ \sum^{N}_{j=1} \left[ \sum^{c}_{y=0} g^{(j)}(y)
\Gamma(y+\eta^{(j)}) \right] } \, .
\end{eqnarray*}

We can also write the unconditional probability of passing the
test as
\begin{eqnarray*}
\PP[ Test \mbox{ }Is \mbox{ }Passed | \vec{x}] & \approx &
\frac{1}{N} \sum_{j=1}^{N} \left[\sum^{c}_{y=0}
\frac{(\lambda^{(j)} T)^y \exp(-\lambda^{(j)} T)}{y!} \right]\\
\mbox{} & \approx & \frac {1}{N} \sum^{N}_{j=1}
\sum^{c}_{y=0} g^{(j)}(y) \Gamma(y+\eta^{(j)})\\
\mbox{} & \approx & \frac {1}{N} \sum^{N}_{j=1}\left[
\sum^{c}_{y=0} \frac
{(\kappa^{(j)})^{\eta^{(j)}}T^{y}\Gamma(y+\eta^{(j)})}{y!
\Gamma(\eta^{(j)})(T+\kappa^{(j)})^{y+\eta^{(j)}}}\right].
\end{eqnarray*}

To obtain a test plan, simultaneously solve the following
pair of nonlinear inequalities:
\begin{equation}\label{ch10:eqn17}
\PP[\lambda\leq\lambda_{0} \con Test\mbox{ }Is\mbox{ }Failed,
\vec{x} ]\leq \alpha
\end{equation}
and
\begin{equation}\label{ch10:eqn18}
\PP[\lambda\geq\lambda_{1} \con Test \mbox{ }Is\mbox{ }Passed,
\vec{x}]\leq\beta,
\end{equation}
where $\lambda_{0} \leq \lambda_{1}$.

Because $T$ is continuous, we can hold either of the risks in
Eqs.~\ref{ch10:eqn17} and \ref{ch10:eqn18} at its precise value.
Holding the posterior producer's risk at exactly $\alpha$, use the
algorithm in Fig. \ref{ch10:fig2} to obtain the desired test plan.
On the other hand, holding the posterior consumer's risk at
precisely $\beta$, simply reverse the two main steps in the
procedure.

\begin{figure}
\setlength{\unitlength}{0.1in} \centering
\begin{picture}(40,40)
\put(0,36){\framebox(5,4){Begin}} \put(2.5,36){\vector(0,-1){4}}
\put(0,28){\framebox(5,4){$c=0$}} \put(2.5,28){\vector(0,-1){4}}
\put(0,16){\framebox(21,8){\shortstack{Posterior Producer's Risk
\\ Solve the nonlinear equation \\ $\PP[\lambda \leq \lambda_0 \con \mbox{Test is Failed}, \vec{x}] = \alpha$ \\ for $T$}}}
\put(2.5,16){\vector(0,-1){4}}
\put(0,8){\framebox(21,4){\shortstack{Posterior Consumer's Risk
\\ $\PP[\lambda \ge \lambda_1 \con \mbox{Test is Passed}, \vec{x}] \le \beta$?}}}
\put(2.5,8){\vector(0,-1){4}} \put(0,0){\framebox(5,4){End}}
\put(0,6){Yes} \put(24,8){\framebox(8,4){$c = c + 1$}}
\put(22,10){No}\put(21,9.5){\vector(1,0){3}}\put(32,10){\line(1,0){2}}
\put(34,10){\line(0,1){10}}\put(34,20){\vector(-1,0){13}}
\end{picture}
\caption{An algorithm for finding Bayesian Poisson test
plans.}\label{ch10:fig2}
\end{figure}

\begin{sidebar}\label{ch10:ex4}
{\bf Hierarchical Poisson test plan for pumps.} \citet{GO87}
provides the data shown in Table~\ref{ch10:tab3} on the number of
pump failures $x$ observed in $t$ thousands of operating hours for
$m = 10$ different systems at the Farley 1 U.S.\ commercial
nuclear power plant.\index[aut]{Gaver, D.}
\index[aut]{O'Muircheartaigh, I.}

\begin{table}
\caption{Pump failure count data from Farley 1 U.S.\ nuclear power
plant (number of failures $x$ in $t$ thousands of operating hours)
\citep{GO87}}\label{ch10:tab3}
 \centering
\begin{tabular}{cccccll}
\hline
\mbox{} & \mbox{} & \multicolumn{1}{c}{$x_{i}$} & \mbox{} & \multicolumn{1}{c}{$t_{i}$} & \mbox{} & \multicolumn{1}{c}{$\widehat{\lambda}$}\\
$\mbox{System}$ & \mbox{} & \multicolumn{1}{c}{(failures)} & \mbox{} & \multicolumn{1}{c}{(thousand hours)} & \mbox{} & \multicolumn{1}{c}{(MLE)}\\
\hline%
1  & \mbox{} & \hspace{0.05in}5  & \mbox{} & \hspace{0.05in}94.320  & \mbox{} & 5.3 x $10^{-2}$\\
2  & \mbox{} & \hspace{0.05in}1  & \mbox{} & \hspace{0.05in}15.720  & \mbox{} & 6.4 x $10^{-2}$\\
3  & \mbox{} & \hspace{0.05in}5  & \mbox{} & \hspace{0.05in}62.880  & \mbox{} & 8.0 x $10^{-2}$\\
4  & \mbox{} & 14 & \mbox{} & 125.760 & \mbox{} & 11.1 x $10^{-2}$\\
5  & \mbox{} & \hspace{0.05in}3  & \mbox{} & \hspace{0.11in}5.240   & \mbox{} & 57.3 x $10^{-2}$\\
6  & \mbox{} & 19 & \mbox{} & \hspace{0.05in}31.440  & \mbox{} & 60.4 x $10^{-2}$\\
7  & \mbox{} &\hspace{0.05in}1   & \mbox{} & \hspace{0.11in}1.048   & \mbox{} & 95.4 x $10^{-2}$\\
8  & \mbox{} & \hspace{0.05in}1  & \mbox{} & \hspace{0.11in}1.048   & \mbox{} & 95.4 x $10^{-2}$\\
9  & \mbox{} & \hspace{0.05in}4  & \mbox{} & \hspace{0.11in}2.096   & \mbox{} & 191.0 x $10^{-2}$\\
10 & \mbox{} & 22 & \mbox{} & \hspace{0.05in}10.480  & \mbox{} & 209.9 x $10^{-2}$\\
\hline
\end{tabular}
\end{table}
Note that we have listed the data in increasing order of the
corresponding maximum likelihood estimates (MLEs)
$\widehat{\lambda}$.

We model the failures as conditionally independent given their
individual failure rates $\lambda_{i}$ with $Poisson(\lambda_i
t_i)$ distributions. Given $\eta$ and $\kappa$, we model the
$\lambda_i$ as i.i.d.\ $Gamma(\eta, \kappa)$ and use independent
and diffuse $InverseGamma(0.001,0.001)$ prior distributions for
$\eta$ and $\kappa$. Table~\ref{ch10:tab4} summarizes the marginal
posterior distributions of $\eta$ and $\kappa$ as well as the
predictive distribution of $\lambda$. The hierarchical Poisson
model fits the pump failure count data well (see
Exercise~\ref{ch4}.\ref{ch4:ex6}).

\begin{table}
\caption{Posterior distribution summaries for the
gamma distribution hyperparameters ($\eta, \kappa$) given
$\vec{x}$ and of the predictive distribution for $\lambda$ for pump example}\label{ch10:tab4}
 \centering
\begin{tabular}{crrrrrrrrrrrrrr}
\hline                              &       &    &       &       &
 &                                & & \multicolumn{3}{c}{Quantiles}  & &
 \\\cline{7-15}
  Parameter&\mbox{}&Mean&\mbox{}&Std
 Dev&\mbox{}\mbox{}\mbox{}
 &\multicolumn{1}{r}{0.025}&\mbox{}&\multicolumn{1}{r}{0.05}&\mbox{}&0.50 & \mbox{} & \multicolumn{1}{r}{0.95}& \mbox{} & \multicolumn{1}{r}{0.975}\\
\hline%
    $\eta$  &\mbox{} & 0.7981&\mbox{} & 0.3635&\mbox{} &        0.2995&\mbox{} &    0.3419&\mbox{} &    0.7272&\mbox{} &    1.4840&\mbox{} &    1.6950 \\
    $\kappa$&\mbox{} &  1.284&\mbox{} &     0.855&\mbox{} &         0.229&\mbox{} &     0.306&\mbox{} &      1.087&\mbox{} &    2.971&\mbox{} &     3.460 \\
    $\lambda$&\mbox{} &     0.796&\mbox{} &     1.306&\mbox{} &         0.002&\mbox{} &     0.007&\mbox{} &      0.390&\mbox{} &    2.939&\mbox{} &     4.037 \\
\hline
\end{tabular}
\end{table}

Using the posterior risk criteria in Eqs.~\ref{ch10:eqn17} and
\ref{ch10:eqn18}, suppose that we want to find the Poisson test
plan\index[sub]{test plan!Poisson} with risk parameters
$\lambda_{0} = 0.2$, $\alpha = 0.05$, $\lambda_{1} = 0.7$, and
$\beta = 0.05$. Using the algorithm in Fig.~\ref{ch10:fig2} with
$N = 10,000$ joint posterior draws of $(\eta, \kappa)$ given
$\vec{x}$, we find the required test plan $T_{0} = 5.43$ and $c =
1$. The actual risks for this test plan are $\PP[\lambda\geq 0.7
\con Test\mbox{ }Is\mbox{ }Passed, \vec{x} ] = 0.0171$ and $\PP[
\lambda\leq 0.2 \con Test\mbox{ }Is\mbox{ } Failed, \vec{x} ] =
0.0500$, and the unconditional probability of passing this test is
approximately 0.44. To implement this test plan for new systems,
accumulate 5,430 hours of pump operating time with repair (or
replacement) of the failed pumps. If no more than one failure
occurs, then the test is passed, otherwise, the test is failed.
\end{sidebar}

Although it is inappropriate for Example~\ref{ch10:ex4}, in many
cases we are free to accumulate the required total test time $T$
by choosing any desired combination of $n$ devices and time on
test $t_{0}$ satisfying $T=nt_{0}$. To illustrate this trade-off,
Fig.~\ref{ch10:fig3} shows selected combinations of the number of
test devices $n$ and required test time $t_{0}$ satisfying $nt_{0}
= 5,430$ hours. For example, if $n = 5$, then test each of these
pumps (with repair or replacement) for $t_{0} = 1,000$ hours each.

\begin{figure}
\centering
\includegraphics[width=0.7\textwidth,angle=270]{Chapter10/ch10_fig3.eps}
\caption{The required test time
$t_{0}$ in hours versus the number of test devices $n$ for a Poisson test plan.}\label{ch10:fig3}
\end{figure}

\section{Weibull Testing}\label{ch10:sec4}\index[sub]{test
plan!Weibull}\index[sub]{assurance
testing}\index[sub]{test!assurance} The previous sections on
binomial and Poisson testing\index[sub]{test
plan!binomial}\index[sub]{test plan!Poisson} considered attribute
test data, which capture the survival/nonsurvival of each device
on test. This section focuses on lifetime data, where testers
record the actual failure times. We assume that the failure times
$t$ follow a $Weibull(\lambda, \beta)$ distribution with scale
parameter $\lambda$ and shape parameter $\beta$, with probability
density function
\begin{equation}\label{ch10:eqn19}
f(t|\lambda,\beta)=\lambda\beta t^{\beta-1}\exp(-\lambda
t^{\beta}), \, \, \, t > 0, \, \lambda > 0, \, \beta>0.
\end{equation}

Suppose that we would like to use the posterior risk criteria to
develop a Weibull test plan $(n,t_0,c)$, where we put $n$ units on
test for $t_0$ time units and the test passes if no more than $c$
units fail. To define the risk criteria, we specify requirements
on reliability at time $t_*$, $R(t_*)$. Let
\begin{eqnarray*}
m(y) & = & (1 - \exp[-\lambda t_0^{\beta}])^{y}
\exp[-(n-y)\lambda t_0^{\beta}] \, ,\\
k_0 & = & -\log(\pi_0)t_*^{-\beta} \quad \mbox{, and}\\
k_1 & = & -\log(\pi_1)t_*^{-\beta} \, .\\
\end{eqnarray*}
For a Weibull test plan\index[sub]{test plan!Weibull} $(n,t_0,c)$,
we calculate the posterior producer's risk as
\begin{eqnarray}\label{ch10:wppr}
\PP(R(t_*) & \geq & \pi_0 \con Test\mbox{ }Is\mbox{ }Failed)\\
\mbox{} & = & \PP(\exp(-\lambda t_*^\beta) \geq \pi_0 \con
Test\mbox{ }Is\mbox{ }Failed)\nonumber\\ \mbox{} & = & \PP(\lambda
\leq -\log(\pi_0)t_*^{-\beta} \con Test\mbox{ }Is\mbox{ }Failed)\nonumber\\
\mbox{} & = & \int_{0}^{\infty} \int_{0}^{k_0} f(\lambda, \beta
\con
Test\mbox{ }Is\mbox{ }Failed) \, \, \D \lambda \D \beta\nonumber\\
\mbox{} & = & \int_{0}^{\infty} \int_{0}^{k_0}
\frac{\PP(Test\mbox{ }Is\mbox{ }Failed \con \lambda, \beta)
p(\lambda, \beta)}{\int_0^{\infty}\int_0^{\infty} \PP(Test\mbox{
}Is\mbox{ }Failed \con \lambda, \beta)p(\lambda,\beta) \, \, \D
\lambda \D \beta} \, \, \D \lambda \D \beta\nonumber\\ \mbox{} & =
& \frac{\int_{0}^{\infty} \int_{0}^{k_0} \left[1 - \sum_{y=0}^{c}
m(y) \right] p(\lambda,\beta) \, \, \D \lambda \D
\beta}{\int_0^{\infty}\int_0^{\infty} \left[1 - \sum_{y=0}^{c}
m(y) \right] p(\lambda,\beta) \, \, \D \lambda \D \beta}\nonumber
\, \, .
\end{eqnarray}
We calculate the posterior consumer's risk as
\begin{eqnarray}\label{ch10:wpcr}
\PP(R(t_*) & \leq & \pi_1 \con Test\mbox{ }Is\mbox{ }Passed)\\
\mbox{} & = & \PP(\exp(-\lambda t_*^\beta) \leq \pi_1 \con
Test\mbox{ }Is\mbox{ }Passed)\nonumber\\ \mbox{} & = & \PP(\lambda
\geq -\log(\pi_1)t_*^{-\beta} \con Test\mbox{ }Is\mbox{ }Passed)\nonumber\\
\mbox{} & = & \int_{0}^{\infty} \int_{k_1}^{\infty} f(\lambda,
\beta \con
Test\mbox{ }Is\mbox{ }Passed) \, \, \D \lambda \D \beta\nonumber\\
\mbox{} & = & \int_{0}^{\infty} \int_{k_1}^{\infty}
\frac{\PP(Test\mbox{ }Is\mbox{ }Passed \con \lambda, \beta)
p(\lambda,\beta)}{\int_0^{\infty}\int_0^{\infty} \PP(Test\mbox{
}Is\mbox{ }Passed \con \lambda, \beta) p(\lambda,\beta) \, \, \D
\lambda \D \beta} \, \, \D \lambda \D \beta\nonumber\\ \mbox{} & =
& \frac{\int_{0}^{\infty} \int_{k_1}^{\infty} \left[\sum_{y=0}^{c}
m(y)] \right] p(\lambda,\beta) \, \, \D \lambda \D
\beta}{\int_0^{\infty}\int_0^{\infty} \left[\sum_{y=0}^{c} m(y)
\right] p(\lambda,\beta) \, \, \D \lambda \D \beta}\nonumber \, \,
.
\end{eqnarray}

\subsection{Single Weibull Population
Testing}\label{ch10:sec4:ss1}\index[sub]{test
plan!Weibull}\index[sub]{assurance
testing}\index[sub]{test!assurance} One description of a failure
time distribution is its reliable life. The \emph{reliable
life}\index[sub]{reliable life}, $t_{R}$, for specified $R$, is
the time beyond which $100 \times R\%$ of the population will
survive. In other words, $t_{R}$ is the $(1 - R)$th quantile of
the failure time distribution. For the Weibull distribution
described by Eq.~\ref{ch10:eqn19}, $t_{R} =
\lambda^{-1/\beta}[-\log (R)]^{1/\beta}$.

Among a variety of testing schemes, we focus here on a minimum
sample size (or zero-failure) test plan $(n,
t_{0},c=0)$\index[sub]{test plan!minimum sample
size}\index[sub]{test plan!zero-failure}. For a zero-failure test
plan, test $n$ devices each for a length of time $t_{0}$, and the
test is passed if we observe no failures. To use such a test plan,
we must determine appropriate values for both $n$ and $t_{0}$.
\citet{ME98} considers such classical test plans in situations
where the Weibull shape parameter $\beta$ is known. We relax this
restriction here by considering plans within a Bayesian
hierarchical framework. In turn, there are two such cases to
study: (1) an assurance test plan based on available data from a
single Weibull population, and (2) an assurance test plan based on
available data from a Weibull accelerated life test
program.\index[aut]{Meeker, W.} \index[aut]{Escobar, L.}

Consider developing a test criterion to assure that $t_{R} >
t_{R^*}$. For example, a manufacturer may want to assure that
$99\%$ of a certain expensive electronic product will survive a
one-year warranty period; in this case, $R = 0.99$ and $t_{R^*} =
8,760$ hours. We use the posterior risk criterion $P[t_{R} >
t_{R^*} \con Test\mbox{ }Is\mbox{ }Passed] \geq 1 - \alpha$. If
the test is passed, we would like a high probability ($1 -
\alpha$) that $t_{R} > t_{R^*}$ --- a high probability that the
0.99 quantile of the lifetime of the electronic products is
greater than 8,760 hours. This leads to the expression
\begin{eqnarray}\label{ch10:wtc}
\PP(t_{R} & > & t_{R^*} \con Test\mbox{ } Is\mbox{ }Passed)\\
\mbox{} & = & \PP(
\lambda^{-1/\beta}[-\log(R)]^{1/\beta} > t_{R^*} \con Test\mbox{ } Is\mbox{ }Passed)\nonumber\\
\mbox{} & = & \PP(\lambda < -\log(R)t_{R^*}^{-\beta} \con
Test\mbox{ } Is\mbox{ }Passed)\nonumber\\
\mbox{} & = & \int_0^{\infty} \int_{0}^{-\log(R)t_{R^*}^{-\beta}}
f(\lambda,\beta \con Test\mbox{ } Is\mbox{
}Passed) \,\, \D \lambda \D \beta\nonumber\\
\mbox{} & = & \int_0^{\infty} \int_{0}^{-\log(R)t_{R^*}^{-\beta}}
\frac{\PP(Test\mbox{ }Is\mbox{ }Passed \con \lambda,
\beta)p(\lambda, \beta)}{\int_{0}^{\infty} \int_{0}^{\infty}
\PP(Test\mbox{
}Is\mbox{ }Passed \con \lambda, \beta)p(\lambda, \beta) \D\lambda \D \beta} \,\, \D \lambda \D \beta\nonumber\\
& = & \frac{\int_0^\infty \int^{-\log(R)t_{R^*}^{-\beta}}_0
\exp(-n\lambda t_0^{\beta}) p(\lambda, \beta) \D \lambda \D
\beta}{\int_0^\infty \int_0^\infty \exp(-n\lambda t_0^{\beta})
p(\lambda, \beta) \D \lambda \D \beta} \geq 1 - \alpha \, .
\nonumber
\end{eqnarray}
Notice the similarities of this risk formulation to the posterior
consumer's risk. As with the Poisson test plan, for a fixed $n$,
we can solve for $t_0$ to meet the desired risk criteria.
With the chosen $c = 0$, this also specifies the level of posterior
producer's risk.

Suppose now that we have failure time data from $m > 1$
situations. Note that some of the available failure time data may
be censored. Let $t_{ij}, i = 1, \ldots, m,  j = 1, \ldots,
n_{i}$, denote the observed failure or censoring time for the
$j$th device in the $i$th situation, and let $\vec{t}$ denote all
the observed failure time data. Given $\lambda_{i}$ and $\beta$,
model the $T_{ij}$ as conditionally independent with $T_{ij} \sim
Weibull(\lambda_{i}, \beta)$. We assume a common shape parameter,
because in practice, the failure times of similar devices often
(but not always) exhibit the same general Weibull shape because
they share common intrinsic failure mechanisms. To complete the
model, let us use the following prior distributions:
\begin{eqnarray*}
\lambda_{i} & \sim & Gamma(\eta, \kappa), \quad i = 1, \ldots, m, \\
(\eta, \kappa) & \sim & p(\eta, \kappa), \mbox{ and}\\
\beta & \sim & p(\beta),
\end{eqnarray*}
with known hyperparameters for $p(\eta, \kappa)$ and $p(\beta)$.

We want to develop a zero-failure test plan for a new situation
where we assume the failure time data will be distributed
$Weibull(\lambda,\beta)$, with $\lambda \sim \Gamma(\eta,
\kappa)$. Conditioning on the observed data $\vec{t}$ in
Eq.~\ref{ch10:wtc},
\begin{eqnarray}\label{ch10:wtc2}
\PP(t_{R} & > & t_{R^*} \con Test\mbox{ } Is\mbox{ }Passed,
\vec{t})\\\nonumber & = & \frac{\int_0^\infty
\int^{-\log(R)t_{R^*}^{-\beta}}_0 \exp(-n\lambda t_0^{\beta})
p(\lambda, \beta \con \vec{t}) \D \lambda \D \beta}{\int_0^\infty
\int_0^\infty \exp(-n\lambda t_0^{\beta}) p(\lambda, \beta \con
\vec{t}) \D \lambda \D \beta} \, .
\end{eqnarray}

Assuming that we have $j = 1, \ldots, N$ MCMC draws from the
posterior distributions (given $\vec{t}$) of $\eta$, $\kappa$, and
$\beta$ and $N$ draws from the predictive distribution of
$\lambda$, $\lambda^{(j)} \sim \Gamma(\eta^{(j)},\kappa^{(j)})$,
we can calculate our criterion as follows:
\begin{eqnarray}\label{ch10:wtc3}
\PP(t_{R} & > & t_{R^*} \con Test\mbox{ } Is\mbox{ }Passed,
\vec{t})\\\nonumber & = & \frac{\int_0^\infty
\int^{-\log(R)t_{R^*}^{-\beta}}_0 \exp(-n\lambda t_0^{\beta})
p(\lambda, \beta \con \vec{t}) \D \lambda \D \beta}{\int_0^\infty
\int_0^\infty \exp(-n\lambda t_0^{\beta}) p(\lambda, \beta \con
\vec{t}) \D \lambda \D \beta}\\\nonumber \mbox{} & \approx &
\frac{\sum_{j=1}^{N} \exp(-n\lambda^{(j)}
t_0^{\beta^{(j)}})I[\lambda^{(j)} \leq
-\log(R)t_{R^*}^{-\beta^{(j)}}]} {\sum_{j=1}^{N}
\exp(-n\lambda^{(j)} t_0^{\beta^{(j)}})}\\\nonumber \mbox{} &
\approx & \frac{\sum^{N}_{j=1} \frac{{(\kappa^{(j)})}^{\eta^{(j)}}
\gamma[\eta^{(j)},[-\log (R)]
(\kappa^{(j)}+nt^{\beta^{(j)}}_{0})/t^{\beta^{(j)}}_{R*}]}{(\kappa^{(j)}+nt^{\beta^{(j)}}_{0})^{\eta^{(j)}}\Gamma(\eta^{(j)})
} }{\sum^{N}_{j=1} \frac{{(\kappa^{(j)})}^{\eta^{(j)}}}
{(\kappa^{(j)}+nt^{\beta^{(j)}}_{0})^{\eta^{(j)}}}}\nonumber \, ,
\end{eqnarray}
where $\gamma(q, z)$ denotes the lower incomplete gamma function.

Note that we may base the choice of number of test devices $n$ on
other considerations, such as cost. It may also be interesting and
useful to see how $t_{0}$ functionally depends on $n$, which we
can examine by varying $n$ over an appropriate range, solving for
$t_{0}$, and plotting the results.

\begin{sidebar}\label{ch10:ex5} {\bf Hierarchical Weibull test plan for pressure
vessels.}\index[sub]{test plan!Weibull} \citet{GK83}  provides the
failure times (in hours) for pressure vessels that were wrapped in
Kevlar-49 fibers and subsequently tested at four different
stresses: 23.4, 25.5, 27.6, and 29.7 megapascals (MPa).
\citet{CKSS91} analyzes these data assuming a constant Weibull
shape parameter. The fibers came from eight different spools
(numbered 1--8) of material, and both studies conclude that there
is a significant spool effect. In this example, consider only the
failure time data obtained at 23.4 MPa. See Table~\ref{ch10:tab5},
which displays the failure time data at all the stresses; an
asterisk indicates a time- or Type I-censored observation.
\index[aut]{Gerstle, F.} \index[aut]{Kunz, S.}\index[aut]{Crowder,
M.} \index[aut]{Kimber, A.} \index[aut]{Smith, R.}
\index[aut]{Sweeting, T.}

\begin{table}
\caption{Failure times of Kevlar-49-wrapped pressure vessels at
four stress levels (An asterisk indicates a time-censored or Type
I-censored observation) \citep{GK83}}\label{ch10:tab5}
% \centering
\begin{tabular}{ccl}\\
\hline Stress (MPa) &  Spool&  \hspace{0.2in} Failure Time (hours)\\
\hline%
29.7& 1  & 444.4\hspace{0.02in} 755.2\hspace{0.02in} 952.2\hspace{0.02in} 1108.2\\
29.7& 2  &  2.2\hspace{0.02in} 8.5\hspace{0.02in} 9.1\hspace{0.02in} 10.2\hspace{0.02in} 22.1\hspace{0.02in} 55.4\hspace{0.02in} 111.4\hspace{0.02in} 158.7\\
29.7& 3  &  12.5\hspace{0.02in} 14.6\hspace{0.02in} 18.7\hspace{0.02in} 101.0\\
29.7& 4  &  254.1\hspace{0.02in} 1148.5\hspace{0.02in} 1569.3\hspace{0.02in} 1750.6\hspace{0.02in} 1802.1\\
29.7& 5  &  8.3\hspace{0.02in} 13.3\hspace{0.02in} 87.5\hspace{0.02in} 243.9\\
29.7& 6  &  6.7\hspace{0.02in} 15.0\hspace{0.02in} 144.0\\
29.7& 7  &  4.0\hspace{0.02in} 4.0\hspace{0.02in} 4.6\hspace{0.02in} 6.1\hspace{0.02in} 7.9\hspace{0.02in} 14.0\hspace{0.02in} 45.9\hspace{0.02in} 61.2\\
29.7& 8  &  98.2\hspace{0.02in} 590.4\hspace{0.02in} 638.2\\

 27.6& 1  &  453.4\hspace{0.02in} 664.5\hspace{0.02in} 930.4\hspace{0.02in} 1755.5\\
 27.6& 2  &  71.2\hspace{0.02in} 199.1\hspace{0.02in} 403.7\hspace{0.02in} 432.2\hspace{0.02in} 514.1\hspace{0.02in} 544.9\hspace{0.02in} 694.1\\
 27.6& 3  &  19.1\hspace{0.02in} 24.3\hspace{0.02in} 69.8\hspace{0.02in} 136.0\\
 27.6& 4  &  876.7\hspace{0.02in} 1275.6\hspace{0.02in} 1536.8\hspace{0.02in} 6177.5\\
 27.6& 5  &  \\
 27.6& 6  &  514.2\hspace{0.02in} 541.6\hspace{0.02in} 1254.9\\
 27.6& 7  &  \\
 27.6& 8  &  554.2\hspace{0.02in} 2046.2\\

 25.5& 1  &  11487.3\hspace{0.02in} 14032.0\hspace{0.02in} 31008.0\\
 25.5& 2  &  1134.2\hspace{0.02in} 1824.3\hspace{0.02in} 1920.1\hspace{0.02in} 2383.0\hspace{0.02in} 3708.9\hspace{0.02in} 5556.0\\
 25.5& 3  &  1087.7\hspace{0.02in} 2442.5\\
 25.5& 4  &  13501.3\hspace{0.02in} 29808.0\\
 25.5& 5  &  11727.1\\
 25.5& 6  &  225.2\hspace{0.02in} 6271.1\hspace{0.02in} 7996.0\\
 25.5& 7  &  503.6\\
 25.5& 8  &  2974.6\hspace{0.02in} 4908.9\hspace{0.02in} 7332.0\hspace{0.02in} 7918.7\hspace{0.02in} 9240.3\hspace{0.02in} 9973.0\\

 23.4& 1  &  41000*\hspace{0.02in} 41000*\hspace{0.02in} 41000*\hspace{0.02in} 41000*\\
 23.4& 2  &  14400.0\\
 23.4& 3  &  8616.0\\
 23.4& 4  &  41000*\hspace{0.02in} 41000*\hspace{0.02in} 41000*\hspace{0.02in} 41000*\\
 23.4& 5  &  9120.0\hspace{0.02in} 20231.0\hspace{0.02in} 35880.0\\
 23.4& 6  &  7320.0\hspace{0.02in} 16104.0\hspace{0.02in} 20233.0\\
 23.4& 7  &  4000.0\hspace{0.02in} 5376.0\\
 23.4& 8  &  41000*\hspace{0.02in} 41000*\hspace{0.02in} 41000* \\

\hline
\end{tabular}
\end{table}

Because any difference in the reliability of the spools is
primarily due to uncontrollable random manufacturing process
variability (or noise), let us model the pressure vessel failure times
corresponding to each spool as conditionally independent with
$Weibull(\lambda_i,\beta)$ distributions. Given $\eta$ and
$\kappa$, the $\lambda_i$ are i.i.d.\ $Gamma(\eta, \kappa)$. In
our analysis of these data, we use independent and diffuse
$InverseGamma(0.01, 0.01)$ prior distributions for $\eta$ and
$\kappa$. Also, we use an independent $Exponential(1.0)$ prior
distribution for $\beta$; the motivation for this prior
distribution is the analysis results of \citet{CKSS91}, which
suggests values of $\beta$ near 1.0.  See Table~\ref{ch10:tab6},
which summarizes the marginal posterior distributions for $\eta$,
$\kappa$, and $\beta$ given $\vec{t}$. The hierarchical Weibull
model fits these data well (see Exercise~\ref{ch4}.\ref{ch4:ex7}).
Table~\ref{ch10:tab6} also summarizes the predictive distribution
of $\lambda$. \index[aut]{Crowder, M.} \index[aut]{Kimber, A.}
\index[aut]{Smith, R.} \index[aut]{Sweeting, T.}

\begin{table}
\caption{Posterior distribution summaries for $\eta$, $\kappa,$
and $\beta$ given $\vec{t}$ and of the predictive distribution for
$\lambda$}\label{ch10:tab6} \centering
\begin{tabular}{crrrrrrrrrrrr}
\hline      &       &    &      &       &
                                          \multicolumn{5}{c}{Quantiles}  &
 \\
 \cline{7-13}
  Parameter&\mbox{}&Mean  &\mbox{}&Std Dev &\mbox{}\mbox{}\mbox{}
 &\multicolumn{1}{r}{0.025}&\mbox{}&\mbox{}&0.50 & \mbox{} &  \mbox{}       & \multicolumn{1}{r}{0.975}\\
\hline
 $\beta$   &\mbox{}& 2.255&\mbox{}&   0.643&\mbox{}&      1.128&\mbox{}&  \mbox{}&  2.211&\mbox{}&   \mbox{}& 3.773\\
    $\eta$  &\mbox{}&0.2100&\mbox{}&    0.1730&\mbox{}&     0.0529&\mbox{}& \mbox{}& 0.1666&\mbox{}&  \mbox{}&    0.6113\\
    $\kappa$    &\mbox{}&2.313E+13&\mbox{}& 3.105E+14&\mbox{}&      7.346E+3&\mbox{}&   \mbox{} &1.121E+8   &\mbox{} &\mbox{} &6.463E+13\\
    $\lambda$   &\mbox{}&5.630E$-$6&\mbox{}&  1.221E$-$4&\mbox{}&       7.304E$-$25&\mbox{}&  \mbox{}&   3.832E$-$11&\mbox{}& \mbox{}&   1.273E$-$5\\
\hline
\end{tabular}
\end{table}


Now suppose that we want to find a Bayesian minimum sample size
test plan\index[sub]{test plan!minimum sample
size}\index[sub]{test plan!zero-failure} at a stress of 23.4 MPa
for $t_{R^*} = 2,000$ hours, $R = 0.9$, and $\alpha = 0.05$. By
letting $n = 1, 2, \ldots, 30$ and solving Eq.~\ref{ch10:wtc3} for
the corresponding test length $t_{0}$, we obtain the graph shown
in Fig.~\ref{ch10:fig4}.

\begin{figure}
\centerline{\includegraphics[width=0.7\textwidth,angle=270]{Chapter10/ex5Figt0VsnV2.ps}}
\caption{The required Weibull test duration $t_{0}$ versus the
number of test devices $n$ for the pressure vessels example.}\label{ch10:fig4}
\end{figure}

For example, suppose that we decide to test $n = 10$ pressure
vessels all wrapped from a particular spool of Kevlar-49 fibers.
Figure~\ref{ch10:fig4} indicates required testing of each of these
pressure vessels for approximately $t_{0} = 1,122$ hours at a
stress of 23.4 MPa. If none of these fail, we can then claim, with
0.95 probability, that at least 90\% of the pressure vessels
wrapped from this spool will survive 2,000 hours at this stress.
\end{sidebar}

\subsection{Combined Weibull Accelerated/Assurance
Testing}\label{ch10:sec4:ss2}\index[sub]{assurance
testing}\index[sub]{test plan!Weibull} Now consider a Bayesian
test plan based on data from a Weibull accelerated test
program\index[sub]{test plan!Weibull!accelerated test}.
Specifically, we consider again the failure time data in
Table~\ref{ch10:tab5}. Let $t_{ij}, i = 1, \ldots, m, j = 1,
\ldots, n_{i}$, denote either the observed failure or censoring
time for the $j$th device from the $i$th situation, with $\vec{t}$
denoting all of the observed failure time data. Let $s_{ij}$ be
the value of the stress $s$ under which we obtained $t_{ij}$.

Conditional on $\lambda_{ij}$ and $\beta$, we model the $T_{ij}$
conditionally independent with a $Weibull(\lambda_{ij}, \beta)$
distribution. Let us define a model for $\lambda_{ij}$ as
\begin{equation}\label{ch10:eqn23}
\lambda_{ij} = \exp(\gamma_{0})s_{ij}^{\gamma_{1}}\omega_{i} \, ,
\end{equation}
where $\omega_{i} > 0$ is the random effect associated with the
$i$th spool, and $\gamma_{0}$ and $\gamma_{1}$ are two regression
parameters used to model the relationship between $\lambda_{ij}$
and $s_{ij}$. Recall that we presented similar regression models
in Chap.~\ref{ch7} (see also \citet{LRAT07}).\index[aut]{Leon,
R.@Le\'{o}n, R.}\index[aut]{Ramachandran, R.}\index[aut]{Ashby,
A.}\index[aut]{Thyagarajan, J.}

We specify prior distributions $\omega_{i} \con \eta, \kappa \sim
Gamma(\eta, \kappa)$, $(\eta, \kappa) \sim p(\eta, \kappa)$,
$\beta \sim p(\beta)$, $\gamma_{0} \sim p(\gamma_{0})$,
$\gamma_{1} \sim p(\gamma_{1})$ with known hyperparameters for
$p(\eta, \kappa)$, $p(\beta)$, $p(\gamma_{0})$, and
$p(\gamma_{1})$.

To develop the Bayesian zero-failure test\index[sub]{test
plan!zero-failure}\index[sub]{test plan!minimum sample size} for a
new spool, we assume the plan consists of testing $n$ samples for
time $t_{0}$ at stress $s_{0}$. We use MCMC to obtain posterior
draws $\eta^{(j)}$, $\kappa^{(j)}$, $\beta^{(j)}$,
$\gamma_{0}^{(j)}$, $\gamma_{1}^{(j)}$ given $\vec{t}$ and
predictive draws $\omega^{(j)} \sim
\Gamma(\eta^{(j)},\kappa^{(j)})$. Let
\begin{eqnarray*}
\bt & = & (\omega, \gamma_0, \gamma_1, \beta) \quad \mbox{and}\\
q^{(j)} & = & \kappa^{(j)}+n
\exp(\gamma_{0}^{(j)})s^{\gamma_{1}^{(j)}}_{0}t^{\beta^{(j)}}_{0}
\, .
\end{eqnarray*}
Our test criterion is calculated as
\begin{eqnarray}\label{ch10:wtc4}
\PP(t_{R} & > & t_{R^*} \con Test\mbox{ } Is\mbox{ }Passed,
\vec{t})\\\nonumber & = & \frac{\int_0^\infty \int_0^\infty
\int_0^\infty
\int_0^{\frac{-\log(R)t_{R^*}^{-\beta}}{\exp(\gamma_0)s_0^{\gamma_1}}}
\exp(-n\exp(\gamma_0)s_0^{\gamma_1}\omega t_0^{\beta}) p(\bt \con
\vec{t}) \D \bt}{\int_0^\infty \int_0^\infty \int_0^\infty
\int_0^\infty \exp(-n\exp(\gamma_0)s_0^{\gamma_1}\omega
t_0^{\beta}) p(\bt \con \vec{t}) \D \bt}\\\nonumber \mbox{} &
\approx & \frac{\sum_{j=1}^{N}
\exp(-n\exp(\gamma_0^{(j)})s_0^{\gamma_1^{(j)}}\omega^{(j)}
t_0^{\beta^{(j)}}) I\left[\omega^{(j)} \leq
\frac{-\log(R)t_{R^*}^{-\beta^{(j)}}}{\exp(\gamma_0^{(j)})s_0^{\gamma_1^{(j)}}}\right]}
{\sum_{j=1}^{N}
\exp(-n\exp(\gamma_0^{(j)})s_0^{\gamma_1^{(j)}}\omega^{(j)}
t_0^{\beta^{(j)}})}\\\nonumber & \approx & \frac{ \sum^{N}_{j=1}
\frac{ {(\kappa^{(j)})}^{\eta^{(j)}} \gamma \left[ \eta^{(j)},
[-\log (R)] \exp(-\gamma_{0}^{(j)})
 s_{0}^{-\gamma_{1}^{(j)}}
 q^{(j)}
/t^{\beta^{(j)}}_{R^{*}} \right]}{\Gamma(\eta^{(j)}){(q^{(j)})
}^{\eta^{(j)}}}} { \sum^{N}_{j=1} \frac
{{(\kappa^{(j)})}^{\eta^{(j)}}}{{{(q^{(j)}) }^{\eta^{(j)}}}}
}\nonumber \, ,
\end{eqnarray}
where $\gamma(q, z)$ denotes the lower incomplete gamma function.

More specifically, to analyze the data in Table~\ref{ch10:tab5},
we use the following independent prior distributions:
\begin{eqnarray*}
\beta & \sim & Exponential(1.0),\\
\eta & \sim &
InverseGamma(0.01, 0.01),\\
\kappa & \sim & InverseGamma(0.01, 0.01),\\
\gamma_{0} & \sim & Normal(0, 10^{6}), \mbox{ and}\\
\gamma_{1} & \sim & Normal(0,10^{6}).
\end{eqnarray*}

See Table~\ref{ch10:tab7}, which summarizes the marginal posterior
distributions for these five parameters. The hierarchical Weibull
regression model fits these data well (see
Exercise~\ref{ch7}.\ref{exercise:goodnessOfFitCh10Example5}).

\begin{table}
\caption{Posterior distribution summaries for $\eta, \kappa,
\beta, \gamma_{0}$, and $\gamma_{1}$ given $\vec{t}$ and
predictive distribution for $\omega$ for pressure vessels
example}\label{ch10:tab7}

 \centering
\begin{tabular}{crrrrrrrrrrrr}
\hline     &             &       &       & &
 & & \multicolumn{3}{c}{Quantiles}  & &
 \\\cline{7-13}
  Parameter&\mbox{}&Mean&\mbox{}&Std
 Dev&\mbox{}\mbox{}\mbox{}
 &\multicolumn{1}{r}{0.025}&\mbox{}&\mbox{}&0.50 & \mbox{} &  \mbox{} & \multicolumn{1}{r}{0.975}\\
\hline%
    $\beta$&\mbox{} &   1.199 &\hspace{0.01in} &    0.085 &\hspace{0.01in} &        1.038 &\hspace{0.01in}  &\hspace{0.01in} &    1.199 &\hspace{0.01in} &    \hspace{0.01in} &   1.3650\\
    $\eta$&\mbox{} &    0.6522 &\hspace{0.01in} &   0.3068 &\hspace{0.01in} &       0.2198 &\hspace{0.01in} &\hspace{0.01in} &   0.5977 &\hspace{0.01in} &   \hspace{0.01in} &    1.391   \\
    $\gamma_0$&\mbox{} &    $-$84.81 &\hspace{0.01in} & 11.31 &\hspace{0.01in} &        $-$105.50 &\hspace{0.01in}  &\hspace{0.01in} &   $-$83.40 &\hspace{0.01in}  &\hspace{0.01in} & $-$66.32 \\
    $\gamma_1$&\mbox{} &    27.84 &\hspace{0.01in} &    1.81 &\hspace{0.01in} &     24.56 &\hspace{0.01in} &\hspace{0.01in} &    27.85 &\hspace{0.01in} &   \hspace{0.01in} &    31.31\\
    $\kappa$&\mbox{} &  3.704E+15 &\hspace{0.01in} &    5.666E+16 &\hspace{0.01in} &        2.403E$-$2 &\hspace{0.01in}  &\hspace{0.01in} &  4.873E+6 &\hspace{0.01in} &\hspace{0.01in} &    7.071E+15\\
$\omega$&\mbox{} & 2.512 &\hspace{0.01in} & 19.42 &\hspace{0.01in}
& 1.34E-17 &\hspace{0.01in}  &\hspace{0.01in} &
3.900E-08 &\hspace{0.01in} & \hspace{0.01in} & 19.73\\
 \hline
\end{tabular}
\end{table}

\iffalse Using $K=5$ bins, the value of the Bayesian $\chi^{2}$
goodness-of-fit test statistic discussed in Sect.~3.4 is computed
to be 4.1.  Comparing this to a $ChiSquared$(4) distribution, we
find that the p-value for testing whether the specified Weibull
regression model with random spool effects ``fits'' the data is
0.30.  Therefore, there is sufficient evidence to suggest that
this model provides an adequate fit to the data. \fi

We can see the effect that stress has on failure time by computing
the posterior distribution of the Weibull mean time to failure
(MTTF) as a function of $s$ for a randomly selected spool of
Kevlar 49. Recall that the MTTF of the Weibull distribution in
Eq.~\ref{ch10:eqn23} is $\lambda^{-1/\beta}\Gamma(1+1/\beta)$,
which, substituting using Eq.~\ref{ch10:eqn23}, becomes
$\exp(-\gamma_{0}/\beta)s^{-\gamma_{1}/\beta}\omega^{-1/\beta}\Gamma(1+1/\beta)$.

In Fig.~\ref{ch10:fig5}, we plot the median log MTTF and a 90\%
central credible interval as a function of stress $s$. Note the
decreasing trend in the Weibull log MTTF as $s$ increases, as well
as the extremely heavy right tail of the posterior log MTTF
distribution for a given value of $s$.

\begin{figure}
\centerline{\includegraphics[width=0.7\textwidth,angle=270]{Chapter10/tenfiveV2.ps}}
\caption{The posterior median and 90\% credible interval of the Weibull log MTTF as a function of stress
$s$ for pressure vessels example.}\label{ch10:fig5}
\end{figure}

We can also see the effect of stress on pressure vessel
reliability by computing the posterior distribution of reliability
as a function of $s$ for a randomly selected Kevlar-49 spool.
Substituting using Eq.~\ref{ch10:eqn23}, the reliability at $t =
2,000$ hours is $\exp[-\exp (\gamma_{0})s^{\gamma_{1}}\omega
(2,000)^{\beta}]$. In Fig.~\ref{ch10:fig6}, we plot the posterior
median and 90\% credible interval for pressure vessel reliability
at $2,000$ hours as a function of stress $s$. Note the significant
increase in the width of the 90\% credible intervals as the stress
$s$ increases.

\begin{figure}
\centerline{\includegraphics[width=0.7\textwidth,angle=270]{Chapter10/tensixV2.ps}}
\caption{The posterior median and 90\% credible interval for $R(2000)$ as a function of stress
$s$ for pressure vessels example.}\label{ch10:fig6}
\end{figure}

Now suppose that we want to find a Bayesian zero-failure test plan
for $t_{R^*} = 2,000$ hours, $R = 0.9$, $\alpha = 0.05$, and
$s_{0} = 23$ MPa. By letting $n = 1, 2, \ldots ,15$ and solving
Eq.~\ref{ch10:wtc4} for the corresponding test length $t_{0}$, we
obtain the solid curve in Fig.~\ref{ch10:fig7}. For example, in
testing $n = 10$ pressure vessels all wrapped from a spool of
Kevlar-49 fiber, Fig.~\ref{ch10:fig7} indicates testing each of
these vessels for approximately $t_{0} = 1,900$ hours. Note that
in Example~\ref{ch10:ex5}, a shorter time of $t_{0} = 1,132$ was
required because the reliability at 2,000 hours was assessed to be
higher based on the only 23 MPa data. If there are no failures,
then we can claim with 0.95 probability that at least 90\% of the
pressure vessels wrapped from this spool will survive 2,000 hours
at a stress of 23 MPa. Figure~\ref{ch10:fig7} also presents the
test length $t_{0}$ as a function of $n$ for stresses $s_{0} = 24$
MPa and $s_{0} = 27$ MPa.

\begin{figure}
\centerline{\includegraphics[width=0.7\textwidth,angle=270]{Chapter10/tensevenV3.ps}}
\caption{The required test duration $t_{0}$ versus the number of
test devices $n$ for three different values of stress $s_{0}$ in
MPa for pressure vessels example. The three stress values are 23
MPa (solid line), 24 MPa (short dashed line), and 27 MPa (long
dashed line).}\label{ch10:fig7}
\end{figure}

From Fig.~\ref{ch10:fig7}, we can make an important observation;
namely, for a given number of devices $n$, as the stress $s_0$
decreases, the required test time $t_{0}$ also decreases. This
desirable situation arises because the failure time data model
using the original accelerated life test data predicts that the
probability of surviving the required 2,000 hours of operation
increases dramatically as the stress decreases. In other words,
because the fitted model predicts long failure times at low
stress, achieving the required assurance needs very little
additional data. For those cases, in which there is high
reliability at low stress, the required assurance is already
embedded in the fitted model. On the other hand, if the
accelerated life test results indicate low reliability at low
stress, then significant assurance testing would be required to
overcome this situation.

\section{Related Reading}\label{ch10:sec5}
There is extensive literature on Bayesian assurance testing dating
back to the late 1960s, and \citet{MW82} describes this early
research. \citet{B86}  clarifies the distinction between a
posterior Bayes' and a modified classical (or average) producer's
risk and compares these two criteria for various test plans.
\index[aut]{Waller, R.}\index[aut]{Brush, G.} \citet{B86} also
highlights the importance of calculating a Bayes' producer's risk
as a supplement to the modified classical producer's
risk.\index[aut]{Brush, G.}\index[aut]{Martz, H.} \citet{SB92}
analyzes the performance of Bayes' and classical assurance test
plans for simultaneously specified consumer's and producer's
risks. Both \citet{B86} and \citet{SB92} conclude that Bayes' and
classical risks need consideration. \index[aut]{Sharma, K.}
\index[aut]{Bhutani, R.}

Since 1990 there have been several more articles concerned with
either Bayesian acceptance sampling or assurance test plans.
\citet{H90},  with comments by \citet{GHMF90}, uses Bayesian
methods to determine a test plan for qualifying the reliability of
some industrial products, whereas \citet{GU90} considers a
Bayesian approach to the assurance testing of highly reliable
devices. \citet{F91} proposes a Bayesian acceptance test plan for
binomial testing, while \citet{SF92} presents a method for
choosing a prior distribution for binomial testing.  In a Master's
thesis, \citet{J91}  develops Bayesian acceptance test plans based
on failure-free life tests and compares these with other test
plans. \citet{PT92}  considers a four-parameter beta prior in
binomial testing. In a somewhat different approach, \citet{MT92}
  uses both quadratic and step-function
loss functions in determining Bayesian acceptance test plans.
\citet{BS93} develops Bayesian sequential reliability
demonstration tests using two different approaches: posterior loss
and predictive loss.  In addition, \citet{BS93}  considers three
test data models. \citet{WYK94}  proposes two different approaches
for integrating life test data into a Bayesian analysis based on
the exponential distribution. For the exponential distribution,
\citet{DK94}  develops Bayesian stopping rules for use in
terminating a sequential assurance test. \citet{V99} considers the
optimization of reliability requirements from a manufacturer's
point of view. \citet{TP03}  proposes Bayesian reliability testing
for a new generation of semiconductor processing equipment, while
\citet{KSB04}  develops reliability demonstration test plans based
on minimizing life cycle costs.\index[aut]{Hart,
L.}\index[aut]{Ganter, W.} \index[aut]{Hart, L.}
\index[aut]{Martz, H.}  \index[aut]{Fisher, R.}\index[aut]{Guess,
F.} \index[aut]{Usher, J.} \index[aut]{Fan, D.} \index[aut]{Sheng,
Z.} \index[aut]{Fan, D.}\index[aut]{Jin, H.}\index[aut]{Pham, T.}
\index[aut]{Turkkan, N.}\index[aut]{Tang,
K.}\index[aut]{Moskowitz, H.}\index[aut]{Berger,
J.}\index[aut]{Berger, J.} \index[aut]{Sun, D.} \index[aut]{Sun,
D.}\index[aut]{Whitmore, G.} \index[aut]{Young, K.}
\index[aut]{Kimber, A.}\index[aut]{Deely, J.} \index[aut]{Keats,
K.}\index[aut]{Vintr, Z.}\index[aut]{Tobias, P.} \index[aut]{Pore,
M.}\index[aut]{Kleyner, A.} \index[aut]{Sandborn, P.}
\index[aut]{Boyle, J.}

\section{Exercises for Chapter~\ref{ch10}}
\begin{enumerate}[\ref{ch10}.1]
\item In determining a Bayesian reliability assurance test plan,
what happens if the prior distribution is especially strong and
satisfies the desired criteria prior to testing?
\begin{enumerate} \item How does one know when this is the case? \item
What calculations should one perform to check whether or not this
is the case?\end{enumerate}

\item \label{ch10:exr1} \begin{enumerate} \item What is the
(classical) producer's risk for a binomial test plan with $n =
15$, $c = 1$, and $\pi_0 = 0.9$? \item What is the (classical)
consumer's risk for a binomial test plan with $n = 15$, $c = 1$,
and $\pi_1 = 0.6$?
\end{enumerate}

\item \label{ch10:exr2} \begin{enumerate} \item What is the
average producer's risk for a binomial test plan with $n = 15$, $c
= 1$, $\pi_0 = 0.9$, and $p(\pi) \sim Beta(10,1)$? \item What is
the average consumer's risk for a binomial test plan with $n =
15$, $c = 1$, $\pi_1 = 0.6$, and $p(\pi) \sim Beta(10,1)$?
\end{enumerate}

\item \label{ch10:exr3} \begin{enumerate} \item What is the
posterior producer's risk for a binomial test plan with $n = 15$,
$c = 1$, $\pi_0 = 0.9$, and $p(\pi) \sim Beta(10,1)$? \item What
is the posterior consumer's risk for a binomial test plan with $n
= 15$, $c = 1$, $\pi_1 = 0.6$, and $p(\pi) \sim Beta(10,1)$?
\end{enumerate}

\item Discuss the similarities and differences between the
producer's and consumer's risks calculated in
Exercises~\ref{ch10}.\ref{ch10:exr1}, \ref{ch10}.\ref{ch10:exr2},
and \ref{ch10}.\ref{ch10:exr3}.

\item Calculate a binomial test plan with a $Uniform(0,1)$ prior
distribution for $\pi$, and $\pi_0 = 0.9$, $\pi_1 = 0.5$, $\alpha
= \beta = 0.05$.

\item \label{exercise:ch10AFW} The auxiliary feedwater (AFW)
system is an important standby safety system in a nuclear power
plant \citep{PGGGK98}. The AFW system probability of starting on demand is an
important indicator of its reliability. The data in
Table~\ref{ch10:tab8} are the number of AFW system failures to
start on demand $x_{i}$ in $n_{i}$ demands at 68 U.S.\ commercial
nuclear power plants. \begin{enumerate} \item Find the Bayesian
hierarchical test plan having the posterior consumer's and
producer's risk values $\pi_{1} = 0.985$, $\beta = 0.05$, $\pi_{0}
= 0.995$, and $\alpha = 0.10$. \item What are the actual posterior
risks when using this test plan? \item What is the unconditional
probability of passing the test when using this test plan? \item
Is there anything unusual about this problem?\end{enumerate}

\begin{table}
\caption{Number of AFW system failures to start on demand $x$
in $n$ demands at 68 U.S.\ commercial nuclear power
plants \citep{PGGGK98}}\label{ch10:tab8}
 \centering
\begin{tabular}{lcr |lcr}
\hline Plant     & $x$                   & $n${\hspace{0.1in}} & {\hspace{0.1in}}Plant            & $x$                   & $n$\\
\hline%
Arkansas       1 & {\hspace{0.1in}}0\mbox{\hspace{0.1in}} & 14\phantom{*}   &{\hspace{0.1in}}North Anna 2     & {\hspace{0.1in}}0\mbox{\hspace{0.1in}} & 18\\
Arkansas       2 & {\hspace{0.1in}}0\mbox{\hspace{0.1in}} & 9\phantom{*}    &{\hspace{0.1in}}Oconee 1         & {\hspace{0.1in}}0\mbox{\hspace{0.1in}} & 18\\
Beaver Valley 1  & {\hspace{0.1in}}0\mbox{\hspace{0.1in}} & 24\phantom{*}   &{\hspace{0.1in}}Oconee 2         & {\hspace{0.1in}}0\mbox{\hspace{0.1in}} & 18\\
Beaver Valley 2  & {\hspace{0.1in}}0\mbox{\hspace{0.1in}} & 43\phantom{*}   &{\hspace{0.1in}}Oconee 3         & {\hspace{0.1in}}0\mbox{\hspace{0.1in}} & 12\\
Braidwood 1      & {\hspace{0.1in}}0\mbox{\hspace{0.1in}} & 13\phantom{*}   &{\hspace{0.1in}}Palisades        & {\hspace{0.1in}}0\mbox{\hspace{0.1in}} & 13 \\
Braidwood 2      & {\hspace{0.1in}}0\mbox{\hspace{0.1in}} & 24\phantom{*}   &{\hspace{0.1in}}Palo Verde 1     & {\hspace{0.1in}}0\mbox{\hspace{0.1in}} & 7\\
Byron 1          & {\hspace{0.1in}}0\mbox{\hspace{0.1in}} & 11\phantom{*}   &{\hspace{0.1in}}Palo Verde 2     & {\hspace{0.1in}}0\mbox{\hspace{0.1in}} & 12\\
Byron 2          & {\hspace{0.1in}}0\mbox{\hspace{0.1in}} & 26\phantom{*}   &{\hspace{0.1in}}Palo Verde 3     & {\hspace{0.1in}}0\mbox{\hspace{0.1in}} & 9\\
Callaway         & {\hspace{0.1in}}0\mbox{\hspace{0.1in}} & 57\phantom{*}   &{\hspace{0.1in}}Point Beach 1    & {\hspace{0.1in}}0\mbox{\hspace{0.1in}} & 8\\
Calvert Cliffs 1 & {\hspace{0.1in}}0\mbox{\hspace{0.1in}} & 12\phantom{*}   &{\hspace{0.1in}}Point Beach 2    & {\hspace{0.1in}}0\mbox{\hspace{0.1in}} & 16\\
Calvert Cliffs 2 & {\hspace{0.1in}}0\mbox{\hspace{0.1in}} & 15\phantom{*}   &{\hspace{0.1in}}Prairie Island 1 & {\hspace{0.1in}}0\mbox{\hspace{0.1in}} & 3\\
Catawba 1        & {\hspace{0.1in}}0\mbox{\hspace{0.1in}} & 41\phantom{*}   &{\hspace{0.1in}}Prairie Island 2 & {\hspace{0.1in}}0\mbox{\hspace{0.1in}} & 7\\
Catawba 2        & {\hspace{0.1in}}0\mbox{\hspace{0.1in}} & 89\phantom{*}   &{\hspace{0.1in}}Robinson 2       & {\hspace{0.1in}}1\mbox{\hspace{0.1in}} & 28\\
Comanche Pk 1    & {\hspace{0.1in}}0\mbox{\hspace{0.1in}} & 66\phantom{*}   &{\hspace{0.1in}}Salem 1          & {\hspace{0.1in}}0\mbox{\hspace{0.1in}} & 24\\
Comanche Pk 2    & {\hspace{0.1in}}0\mbox{\hspace{0.1in}} & 14\phantom{*}   &{\hspace{0.1in}}Salem 2          & {\hspace{0.1in}}0\mbox{\hspace{0.1in}} & 32\\
Cook 1           & {\hspace{0.1in}}0\mbox{\hspace{0.1in}} & 18\phantom{*}   &{\hspace{0.1in}}San Onofre 2     & {\hspace{0.1in}}0\mbox{\hspace{0.1in}} & 13 \\
Cook 2           & {\hspace{0.1in}}0\mbox{\hspace{0.1in}} & 36\phantom{*}   &{\hspace{0.1in}}San Onofre 3     & {\hspace{0.1in}}0\mbox{\hspace{0.1in}} & 17\\
Crystal River 3  & {\hspace{0.1in}}1\mbox{\hspace{0.1in}} & 16\phantom{*}   &{\hspace{0.1in}}Seabrook         & {\hspace{0.1in}}0\mbox{\hspace{0.1in}} & 17\\
Diablo Canyon 1  & {\hspace{0.1in}}0\mbox{\hspace{0.1in}} & 46\phantom{*}   &{\hspace{0.1in}}Sequoyah 1       & {\hspace{0.1in}}0\mbox{\hspace{0.1in}} & 30\\
Diablo Canyon 2  & {\hspace{0.1in}}0\mbox{\hspace{0.1in}} & 30\phantom{*}   &{\hspace{0.1in}}Sequoyah 2       & {\hspace{0.1in}}0\mbox{\hspace{0.1in}} & 41\\
Farley 1         & {\hspace{0.1in}}0\mbox{\hspace{0.1in}} & 34\phantom{*}   &{\hspace{0.1in}}South Texas 1    & {\hspace{0.1in}}0\mbox{\hspace{0.1in}} & 69\\
Farley 2         & {\hspace{0.1in}}0\mbox{\hspace{0.1in}} & 54\phantom{*}   &{\hspace{0.1in}}South Texas 2    & {\hspace{0.1in}}0\mbox{\hspace{0.1in}} & 87\\
Fort Calhoun     & {\hspace{0.1in}}0\mbox{\hspace{0.1in}} & 5\phantom{*}    &{\hspace{0.1in}}St. Lucie 1      & {\hspace{0.1in}}0\mbox{\hspace{0.1in}} & 35\\
Ginna            & {\hspace{0.1in}}0\mbox{\hspace{0.1in}} & 28\phantom{*}   &{\hspace{0.1in}}St. Lucie 2      & {\hspace{0.1in}}0\mbox{\hspace{0.1in}} & 21\\
Harris           & {\hspace{0.1in}}0\mbox{\hspace{0.1in}} & 98\phantom{*}   &{\hspace{0.1in}}Summer           & {\hspace{0.1in}}0\mbox{\hspace{0.1in}} & 24 \\
Indian Point 2   & {\hspace{0.1in}}1\mbox{\hspace{0.1in}} & 24\phantom{*}   &{\hspace{0.1in}}Surry 1          & {\hspace{0.1in}}0\mbox{\hspace{0.1in}} & 26\\
Indian Point 3   & {\hspace{0.1in}}2\mbox{\hspace{0.1in}} & 32\phantom{*}   &{\hspace{0.1in}}Surry 2          & {\hspace{0.1in}}0\mbox{\hspace{0.1in}} & 32\\
Kewaunee         & {\hspace{0.1in}}0\mbox{\hspace{0.1in}} & 26\phantom{*}   &{\hspace{0.1in}}Three Mile Isl 1 & {\hspace{0.1in}}0\mbox{\hspace{0.1in}} & 6\\
Maine Yankee     & {\hspace{0.1in}}0\mbox{\hspace{0.1in}} & 23\phantom{*}   &{\hspace{0.1in}}Vogtle 1         & {\hspace{0.1in}}0\mbox{\hspace{0.1in}} & 103\\
McGuire 1        & {\hspace{0.1in}}0\mbox{\hspace{0.1in}} & 45\phantom{*}   &{\hspace{0.1in}}Vogtle 2         & {\hspace{0.1in}}0\mbox{\hspace{0.1in}} & 45\\
McGuire 2        & {\hspace{0.1in}}0\mbox{\hspace{0.1in}} & 44\phantom{*}   &{\hspace{0.1in}}Waterford 3      & {\hspace{0.1in}}0\mbox{\hspace{0.1in}} & 38\\
Millstone  2     & {\hspace{0.1in}}1\mbox{\hspace{0.1in}} & 11\phantom{*}   &{\hspace{0.1in}}Wolf Creek       & {\hspace{0.1in}}0\mbox{\hspace{0.1in}} & 51\\
Millstone  3     & {\hspace{0.1in}}0\mbox{\hspace{0.1in}} & 54\phantom{*}   &{\hspace{0.1in}}Zion 1           & {\hspace{0.1in}}0\mbox{\hspace{0.1in}} & 13\\
North Anna 1     & {\hspace{0.1in}}0\mbox{\hspace{0.1in}} & 20\phantom{*}   &{\hspace{0.1in}}Zion 2           & {\hspace{0.1in}}0\mbox{\hspace{0.1in}} & 8\\
\hline
\end{tabular}
\end{table}


\item Derive Eq.~\ref{ch10:eqn14}.

\item \label{ch10:hw4} \citet{B62} provides data that apparently
originated at the U.S.\ Bureau of Naval Weapons regarding the
number of minor, major, and critical defectives in successive
MIL-STD-105B samples of some material. The data in
Table~\ref{ch10:tab9} consist of the observed frequencies of the
number of minor defectives $x$ in samples of size $n = 150$ from
$m = 205$ lots each of size 2016 items of this material.
\begin{enumerate} \item Using the ``hybrid'' posterior consumer's and average
producer's risk criteria, find the Bayesian hierarchical test plan
having the risk values $\beta = 0.10$ and $\alpha = 0.05$ for
$\pi^* = 0.975$. \item What are the actual risks when using this
test plan? \item What is the unconditional probability of passing
this test?\end{enumerate} \index[aut]{Borg, J.}


\begin{table}
\caption{Minor defectives from MIL-STD-105B sampling of material
\citep{B62}}\label{ch10:tab9}
 \centering
\begin{tabular}{cccc|ccc}
\hline
$x$ & \mbox{} & Frequency & \mbox{} & $x$ & \mbox{} & Frequency\\
\hline%
0 & \mbox{} & 68 & \mbox{} & \hspace{0.05in}9  & \mbox{} & 2\\
1 & \mbox{} & 45 & \mbox{} & 10 & \mbox{} & 1\\
2 & \mbox{} & 24 & \mbox{} & 12 & \mbox{} & 1\\
3 & \mbox{} & 20 & \mbox{} & 13 & \mbox{} & 1\\
4 & \mbox{} & \hspace{0.05in}8  & \mbox{} & 15 & \mbox{} & 4\\
5 & \mbox{} & \hspace{0.05in}7  & \mbox{} & 18 & \mbox{} & 1\\
6 & \mbox{} & \hspace{0.05in}8  & \mbox{} & 20 & \mbox{} & 1\\
7 & \mbox{} & 10 & \mbox{} & 22 & \mbox{} & 1\\
8 & \mbox{}\ \ \ \ \  & \hspace{0.05in}3  & \mbox{} &  & \mbox{} & \\
\hline
\end{tabular}
\end{table}


\item Using the data in Exercise~\ref{ch10}.\ref{ch10:hw4}, find
the Bayesian hierarchical test plan having the posterior
consumer's and producer's risk values $\pi_{1} = 0.96$, $\beta =
0.10$, $\pi_{0} = 0.975$, and $\alpha = 0.05$.
\begin{enumerate} \item What are the actual risks when using this test
plan? \item What is the unconditional probability of passing this
test?\end{enumerate}

\item \label{ch10:exr4} \begin{enumerate} \item What is the
posterior producer's risk for a Poisson test plan with $T = 10$,
$c = 3$, $\lambda_0 = 3.0$, and $p(\lambda) \sim Gamma(5,1)$?
\item What is the posterior consumer's risk for a Poisson test
plan with $T = 10$, $c = 3.0$, $\lambda_1 = 7.0$, and $p(\lambda)
\sim Gamma(5,1)$?
\end{enumerate}

\item \label{ch10:hw6} For Poisson testing
presented in Sect.~\ref{ch10:sec3}, show that
\begin{displaymath}
\PP(Test\mbox{ }Is\mbox{
}Failed|\lambda\leq\lambda_{0},\eta,\kappa) =
1-\frac{\kappa^{\eta}\sum^{c}_{x=0}
\frac{T_{0}^{x}\gamma[x+\eta,(T_{0}+\kappa)\lambda_{0}]}
{x!(T_{0}+\kappa)^{x+\eta}}}{\gamma(\eta,\kappa\lambda_{0})} \, .
\end{displaymath}

\item Using the expression given in
Exercise~\ref{ch10}.\ref{ch10:hw6} and the pump failure data in
Table~\ref{ch10:tab3}, find the Bayesian hierarchical test plan
having the ``hybrid'' posterior consumer's and average producer's
risk values $\lambda_{1} = 0.3$, $\beta = 0.10$, $\lambda_{0}=
0.2$, and $\alpha = 0.05$. \begin{enumerate} \item What are the
actual risks for this test plan? \item What is the unconditional
probability of passing this test?\end{enumerate}

\item Using the pump failure data in Table~\ref{ch10:tab3}, find
the Bayesian hierarchical test plan having the posterior
consumer's and producer's risk values $\lambda_{1} = 0.1$, $\beta
= 0.10$, $\lambda_{0}= 0.05$, and $\alpha = 0.05$.
\begin{enumerate} \item What are the actual risks for this test plan? \item What is
the unconditional probability of passing this test? \item Is this
a good test plan to use?\end{enumerate}

\item \label{ch10:ex9} For the model in Sect.~\ref{ch10:sec4:ss1},
show that we may approximate the unconditional probability of
passing the Bayesian minimum sample size test plan by
\begin{displaymath}
 \PP[Test\mbox{ }Is\mbox{ }Passed \con \vec{t} ] \approx \frac
 {1}{N}\sum^{N}_{j=1}\left(\frac
 {\kappa^{(j)}}{\kappa^{(j)}+nt^{\beta^{(j)}}_{0}}\right)^{\eta^{(j)}}.
\end{displaymath}

\item Using the expression in Exercise~\ref{ch10}.\ref{ch10:ex9},
what is the approximate unconditional probability of passing the
Bayesian test plan ($n = 10, t_{0} = 1,122$) given in
Example~\ref{ch10:ex5}?

\item \citet{GK83}  gives the failure times for Kevlar-49-wrapped
pressure vessels at a stress of 25.5 MPa. Table~\ref{ch10:tab10}
displays these data. For $t_{R^*} = 300$ hours, $R = 0.9$, $\alpha
= 0.05$, and these Weibull distributed data, find the Bayesian
minimum sample size test plan time $t_{0}$ that we must test each
of $n = 5$ items. What is the unconditional probability of passing
this test? Is this a satisfactory test plan? \index[aut]{Gerstle,
F.} \index[aut]{Kunz, S.}


\begin{table}
\caption{Failure times of Kevlar-49-wrapped pressure vessels at a
stress of 25.5 MPa \citep{GK83}}\label{ch10:tab10}
 \centering
\begin{tabular}{cll}
\hline Spool & \mbox{} & \hspace{0.5in}Failure Time (hours) \\
\hline%
1 & \mbox{} & 11487.3, 14032.0, 31008.0\\
2 & \mbox{} & 1134.3, 1824.3, 1920.1, 2383.0,
 3708.9, 5556.0\\
3 & \mbox{} & 1087.7, 2442.5\\
4 & \mbox{} & 13501.3, 29808.0\\
5 & \mbox{} & 11727.1\\
6 & \mbox{} & 225.2, 6271.1, 7996.0\\
7 & \mbox{} & 503.6\\
8 & \mbox{} & 2974.6, 4908.9, 7332.0, 7918.7,
 9240.3, 9973.0\\
\hline
\end{tabular}
\end{table}

\item For the Weibull testing described in
Sect.~\ref{ch10:sec4}, suppose that we want to find a Bayesian
minimum sample size test plan to assure that, at some
specified time $t_{R}$, the Weibull reliability $R$ is at least as
large as a requirement $R^*$ at the $100\times (1 - \alpha)$\%
credible level. How does this test plan compare to the one based on
the reliable life criterion?
\end{enumerate}

\end{document}
